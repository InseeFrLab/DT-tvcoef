% Options for packages loaded elsewhere
\PassOptionsToPackage{unicode}{hyperref}
\PassOptionsToPackage{hyphens}{url}
\PassOptionsToPackage{dvipsnames,svgnames,x11names}{xcolor}
%
\documentclass[
  a4paper,
  DIV=11,
  numbers=noendperiod,
  french]{scrartcl}

\usepackage{amsmath,amssymb}
\usepackage{iftex}
\ifPDFTeX
  \usepackage[T1]{fontenc}
  \usepackage[utf8]{inputenc}
  \usepackage{textcomp} % provide euro and other symbols
\else % if luatex or xetex
  \usepackage{unicode-math}
  \defaultfontfeatures{Scale=MatchLowercase}
  \defaultfontfeatures[\rmfamily]{Ligatures=TeX,Scale=1}
\fi
\usepackage{lmodern}
\ifPDFTeX\else  
    % xetex/luatex font selection
\fi
% Use upquote if available, for straight quotes in verbatim environments
\IfFileExists{upquote.sty}{\usepackage{upquote}}{}
\IfFileExists{microtype.sty}{% use microtype if available
  \usepackage[]{microtype}
  \UseMicrotypeSet[protrusion]{basicmath} % disable protrusion for tt fonts
}{}
\makeatletter
\@ifundefined{KOMAClassName}{% if non-KOMA class
  \IfFileExists{parskip.sty}{%
    \usepackage{parskip}
  }{% else
    \setlength{\parindent}{0pt}
    \setlength{\parskip}{6pt plus 2pt minus 1pt}}
}{% if KOMA class
  \KOMAoptions{parskip=half}}
\makeatother
\usepackage{xcolor}
\setlength{\emergencystretch}{3em} % prevent overfull lines
\setcounter{secnumdepth}{5}
% Make \paragraph and \subparagraph free-standing
\ifx\paragraph\undefined\else
  \let\oldparagraph\paragraph
  \renewcommand{\paragraph}[1]{\oldparagraph{#1}\mbox{}}
\fi
\ifx\subparagraph\undefined\else
  \let\oldsubparagraph\subparagraph
  \renewcommand{\subparagraph}[1]{\oldsubparagraph{#1}\mbox{}}
\fi

\usepackage{color}
\usepackage{fancyvrb}
\newcommand{\VerbBar}{|}
\newcommand{\VERB}{\Verb[commandchars=\\\{\}]}
\DefineVerbatimEnvironment{Highlighting}{Verbatim}{commandchars=\\\{\}}
% Add ',fontsize=\small' for more characters per line
\usepackage{framed}
\definecolor{shadecolor}{RGB}{241,243,245}
\newenvironment{Shaded}{\begin{snugshade}}{\end{snugshade}}
\newcommand{\AlertTok}[1]{\textcolor[rgb]{0.68,0.00,0.00}{#1}}
\newcommand{\AnnotationTok}[1]{\textcolor[rgb]{0.37,0.37,0.37}{#1}}
\newcommand{\AttributeTok}[1]{\textcolor[rgb]{0.40,0.45,0.13}{#1}}
\newcommand{\BaseNTok}[1]{\textcolor[rgb]{0.68,0.00,0.00}{#1}}
\newcommand{\BuiltInTok}[1]{\textcolor[rgb]{0.00,0.23,0.31}{#1}}
\newcommand{\CharTok}[1]{\textcolor[rgb]{0.13,0.47,0.30}{#1}}
\newcommand{\CommentTok}[1]{\textcolor[rgb]{0.37,0.37,0.37}{#1}}
\newcommand{\CommentVarTok}[1]{\textcolor[rgb]{0.37,0.37,0.37}{\textit{#1}}}
\newcommand{\ConstantTok}[1]{\textcolor[rgb]{0.56,0.35,0.01}{#1}}
\newcommand{\ControlFlowTok}[1]{\textcolor[rgb]{0.00,0.23,0.31}{#1}}
\newcommand{\DataTypeTok}[1]{\textcolor[rgb]{0.68,0.00,0.00}{#1}}
\newcommand{\DecValTok}[1]{\textcolor[rgb]{0.68,0.00,0.00}{#1}}
\newcommand{\DocumentationTok}[1]{\textcolor[rgb]{0.37,0.37,0.37}{\textit{#1}}}
\newcommand{\ErrorTok}[1]{\textcolor[rgb]{0.68,0.00,0.00}{#1}}
\newcommand{\ExtensionTok}[1]{\textcolor[rgb]{0.00,0.23,0.31}{#1}}
\newcommand{\FloatTok}[1]{\textcolor[rgb]{0.68,0.00,0.00}{#1}}
\newcommand{\FunctionTok}[1]{\textcolor[rgb]{0.28,0.35,0.67}{#1}}
\newcommand{\ImportTok}[1]{\textcolor[rgb]{0.00,0.46,0.62}{#1}}
\newcommand{\InformationTok}[1]{\textcolor[rgb]{0.37,0.37,0.37}{#1}}
\newcommand{\KeywordTok}[1]{\textcolor[rgb]{0.00,0.23,0.31}{#1}}
\newcommand{\NormalTok}[1]{\textcolor[rgb]{0.00,0.23,0.31}{#1}}
\newcommand{\OperatorTok}[1]{\textcolor[rgb]{0.37,0.37,0.37}{#1}}
\newcommand{\OtherTok}[1]{\textcolor[rgb]{0.00,0.23,0.31}{#1}}
\newcommand{\PreprocessorTok}[1]{\textcolor[rgb]{0.68,0.00,0.00}{#1}}
\newcommand{\RegionMarkerTok}[1]{\textcolor[rgb]{0.00,0.23,0.31}{#1}}
\newcommand{\SpecialCharTok}[1]{\textcolor[rgb]{0.37,0.37,0.37}{#1}}
\newcommand{\SpecialStringTok}[1]{\textcolor[rgb]{0.13,0.47,0.30}{#1}}
\newcommand{\StringTok}[1]{\textcolor[rgb]{0.13,0.47,0.30}{#1}}
\newcommand{\VariableTok}[1]{\textcolor[rgb]{0.07,0.07,0.07}{#1}}
\newcommand{\VerbatimStringTok}[1]{\textcolor[rgb]{0.13,0.47,0.30}{#1}}
\newcommand{\WarningTok}[1]{\textcolor[rgb]{0.37,0.37,0.37}{\textit{#1}}}

\providecommand{\tightlist}{%
  \setlength{\itemsep}{0pt}\setlength{\parskip}{0pt}}\usepackage{longtable,booktabs,array}
\usepackage{calc} % for calculating minipage widths
% Correct order of tables after \paragraph or \subparagraph
\usepackage{etoolbox}
\makeatletter
\patchcmd\longtable{\par}{\if@noskipsec\mbox{}\fi\par}{}{}
\makeatother
% Allow footnotes in longtable head/foot
\IfFileExists{footnotehyper.sty}{\usepackage{footnotehyper}}{\usepackage{footnote}}
\makesavenoteenv{longtable}
\usepackage{graphicx}
\makeatletter
\def\maxwidth{\ifdim\Gin@nat@width>\linewidth\linewidth\else\Gin@nat@width\fi}
\def\maxheight{\ifdim\Gin@nat@height>\textheight\textheight\else\Gin@nat@height\fi}
\makeatother
% Scale images if necessary, so that they will not overflow the page
% margins by default, and it is still possible to overwrite the defaults
% using explicit options in \includegraphics[width, height, ...]{}
\setkeys{Gin}{width=\maxwidth,height=\maxheight,keepaspectratio}
% Set default figure placement to htbp
\makeatletter
\def\fps@figure{htbp}
\makeatother

\DeclareMathOperator{\argmin}{argmin}
\DeclareMathOperator{\argmax}{argmax}


\newcommand\1{{\mathds 1}}
\newcommand\ud{\,\mathrm{d}}
\newcommand{\transp}[1]{{}^t\!#1}
\newcommand{\bf}[1]{{\boldsymbol #1}}
\newcommand{\E}[1]{\mathbb{E}\left[#1\right]}

\usepackage{mathrsfs, dsfont}
\usepackage[style=authoryear,uniquename=false, uniquelist=false]{biblatex}
\DefineBibliographyStrings{french}{andothers={et\addabbrvspace alii}}
\usepackage{booktabs}
\usepackage{caption}
\usepackage{longtable}
\usepackage{colortbl}
\usepackage{array}
\KOMAoption{captions}{tableheading,figureheading}
\makeatletter
\@ifpackageloaded{caption}{}{\usepackage{caption}}
\AtBeginDocument{%
\ifdefined\contentsname
  \renewcommand*\contentsname{Table des matières}
\else
  \newcommand\contentsname{Table des matières}
\fi
\ifdefined\listfigurename
  \renewcommand*\listfigurename{Liste des Figures}
\else
  \newcommand\listfigurename{Liste des Figures}
\fi
\ifdefined\listtablename
  \renewcommand*\listtablename{Liste des Tables}
\else
  \newcommand\listtablename{Liste des Tables}
\fi
\ifdefined\figurename
  \renewcommand*\figurename{Figure}
\else
  \newcommand\figurename{Figure}
\fi
\ifdefined\tablename
  \renewcommand*\tablename{Table}
\else
  \newcommand\tablename{Table}
\fi
}
\@ifpackageloaded{float}{}{\usepackage{float}}
\floatstyle{ruled}
\@ifundefined{c@chapter}{\newfloat{codelisting}{h}{lop}}{\newfloat{codelisting}{h}{lop}[chapter]}
\floatname{codelisting}{Listing}
\newcommand*\listoflistings{\listof{codelisting}{Liste des Listings}}
\usepackage{amsthm}
\theoremstyle{remark}
\AtBeginDocument{\renewcommand*{\proofname}{Preuve}}
\newtheorem*{remark}{Remarque}
\newtheorem*{solution}{Solution}
\newtheorem{refremark}{Remarque}[section]
\newtheorem{refsolution}{Solution}[section]
\makeatother
\makeatletter
\makeatother
\makeatletter
\@ifpackageloaded{caption}{}{\usepackage{caption}}
\@ifpackageloaded{subcaption}{}{\usepackage{subcaption}}
\makeatother
\makeatletter
\@ifpackageloaded{fontawesome5}{}{\usepackage{fontawesome5}}
\makeatother
\ifLuaTeX
\usepackage[bidi=basic]{babel}
\else
\usepackage[bidi=default]{babel}
\fi
\babelprovide[main,import]{french}
% get rid of language-specific shorthands (see #6817):
\let\LanguageShortHands\languageshorthands
\def\languageshorthands#1{}
\ifLuaTeX
  \usepackage{selnolig}  % disable illegal ligatures
\fi
\usepackage[style=authoryear,]{biblatex}
\addbibresource{biblio.bib}
\usepackage{bookmark}

\IfFileExists{xurl.sty}{\usepackage{xurl}}{} % add URL line breaks if available
\urlstyle{same} % disable monospaced font for URLs
\hypersetup{
  pdftitle={Utilisation de modèles de régression à coefficients variant dans le temps pour la prévision conjoncturelle},
  pdfauthor={Alain Quartier-la-Tente},
  pdflang={fr},
  colorlinks=true,
  linkcolor={blue},
  filecolor={Maroon},
  citecolor={Blue},
  urlcolor={Blue},
  pdfcreator={LaTeX via pandoc}}

\title{Utilisation de modèles de régression à coefficients variant dans
le temps pour la prévision conjoncturelle}
\author{Alain Quartier-la-Tente}
\date{}

\begin{document}
\maketitle

\renewcommand{\thepage}{\roman{page}}

\subsubsection*{Résumé}\label{ruxe9sumuxe9}
\addcontentsline{toc}{subsubsection}{Résumé}

Cette étude décrit trois méthodes d'estimation de modèles de régression
linéaire avec des coefficients variant dans le temps : régression par
morceaux, régression locale et régression avec coefficients
stochastiques (modélisation espace-état). Elle détaille également leur
implémentation sous R grâce au package \texttt{tvCoef}. À travers une
analyse comparative sur une trentaine de modèles de prévision
trimestrielle, nous montrons que l'utilisation de ces méthodes,
notamment par la modélisation espace-état, réduit les erreurs de
prévision lorsque des ruptures sont présentes dans les coefficients. Par
ailleurs, même lorsque les tests classiques concluent à la constance des
coefficients, la régression avec coefficients stochastiques peut
permettre de réduire les erreurs de prévision. Cependant, les
incertitudes liées à l'estimation de certains hyperparamètres peuvent
augmenter les erreurs de prévision en temps réel, en particulier pour la
régression locale. Ainsi, une analyse économique des paramètres estimés
demeure essentielle.

Cette étude est entièrement reproductible et tous les codes utilisés
sont disponibles sous \url{https://github.com/InseeFrLab/DT-tvcoef}.

Mots clés : séries temporelles, prévisions, séries longues.

\subsubsection*{Abstract}\label{abstract}
\addcontentsline{toc}{subsubsection}{Abstract}

This study describes three methods for estimating linear regression
models with time-varying coefficients: piecewise regression, local
regression, and regression with stochastic coefficients (state-space
modeling). It also details their implementation in R using the
\texttt{tvCoef} package. Through a comparative analysis of around thirty
quarterly forecasting models, we show that the use of these methods,
especially thanks to the state-space modeling, reduces forecast errors
when breakpoints are present in the coefficients. Moreover, even when
traditional tests conclude that the coefficients are stable, regression
with stochastic coefficients can still help reduce forecast errors.
However, uncertainties related to estimating certain hyperparameters can
increase real-time forecast errors, especially for local regression.
Thus, an economic analysis of estimated parameters remains essential.

This study is fully reproducible and all the codes used are available
under \url{https://github.com/InseeFrLab/DT-tvcoef}.

Keywords: time series, forecast, long time series.

JEL Classification: C22, C53.

\newpage
\renewcommand*\contentsname{Table des matières}
{
\hypersetup{linkcolor=}
\setcounter{tocdepth}{2}
\tableofcontents
}
\pagenumbering{arabic}
\newpage

\section{Introduction}\label{introduction}

De nombreux modèles de prévision s'appuient sur l'hypothèse que les
relations entre les variables sont fixes dans le temps. C'est par
exemple le cas des modèles de régressions linéaires utilisés couramment
dans la statistique publique. Ainsi, les producteurs de séries
désaisonnalisées appliquent des modèles RegARIMA pour la correction des
effets de calendrier et les comptes nationaux trimestriels utilisent des
modèles d'étalonnage-calage pour caler les séries sur les comptes
nationaux annuels. Pour la prévision des grands agrégats
macroéconomiques, l'Insee \autocite[e.g.,][]{ndc2015prev} et la Banque
de France \autocite[e.g.,][]{OPTIM} utilisent notamment des modèles de
régression linéaire pour prévoir la croissance et le modèle
macroéconomique Mésange \autocite{mesange} s'appuie sur des modèles à
correction d'erreur pour modéliser les comportements macroéconomiques.
Ces méthodes fournissent généralement de bons résultats et ont
l'avantage d'être facilement interprétables. Cependant, même si
l'hypothèse de stabilité des coefficients peut avoir du sens sur courte
période, elle n'est généralement plus vérifiée lorsque les modèles sont
estimés sur longue période, ce qui conduit à des modèles sous-optimaux.

Pour palier ce problème, une solution simple consiste à utiliser moins
de données pour estimer les modèles. Par exemple, le guide des bonnes
pratiques sur l'ajustement saisonnier \autocite{eurostat2015guidelines}
recommande de ne pas désaisonnaliser des séries de plus de 20 ans.
Toutefois, cela conduit à perdre l'historique des données et
l'information que l'on peut en tirer et ne résout pas le problème
lorsque la rupture est récente. Par ailleurs, comme montré par
\textcite{JMS2018} pour la désaisonnalisation des séries d'indice de
production industrielle, lorsqu'il faut analyser les modèles sur
l'ensemble de la période (par exemple dans le cadre de la correction des
jours ouvrables), il est nécessaire de mettre en place des méthodes de
chaînage afin de prendre en compte la rupture introduite par
l'utilisation de plusieurs modèles. Ainsi, dans certains cas il peut
être préférable d'utiliser des modèles qui prennent directement en
compte les ruptures.

Cette étude s'intéresse à différentes méthodes d'estimation de
coefficients variant dans le temps dans le cadre de la prévision
conjoncturelle. Ces méthodes se regroupent en trois catégories : les
modèles de régression par morceaux, les régressions locales et les
régressions avec coefficients stochastiques (estimés par une
modélisation espace-état). La première suppose l'existence d'une rupture
brutale sur les coefficients à une certaine date ; les deux autres
supposent que les coefficients évoluent progressivement dans le temps
sans existence de rupture brutale. Pour simplifier l'implémentation de
ces méthodes, ainsi que leur comparaison, le package R \texttt{tvCoef}
(\url{https://github.com/InseeFrLab/tvCoef}) a également été développé
lors de cette étude. Cette étude est entièrement reproductible et tous
les codes utilisés sont disponibles sous
\url{https://github.com/InseeFrLab/DT-tvcoef}.

Après une description de deux tests permettant de tester si les
coefficients sont fixes dans le temps (section~\ref{sec-tests}), nous
décrivons trois méthodes pour estimer des coefficients variant dans le
temps et montrons comment les implémenter à partir d'un modèle de
prévision de la croissance du PIB français
(section~\ref{sec-desc-meth}). Nous comparons ensuite les qualités
prédictives des différentes méthodes sur une trentaine de modèles de
prévision trimestrielle (section~\ref{sec-comp-generales}). Nous
montrons que, lorsque l'hypothèse de constance des coefficients n'est
pas vérifiée, l'utilisation de ces modèles (notamment la régression avec
coefficients stochastiques) permet de réduire les erreurs de prévision.
Par ailleurs, même lorsque les tests classiques concluent à la constance
des coefficients, la régression avec coefficients stochastiques peut
permettre de réduire les erreurs de prévision.

\section{Modélisation générale et tests}\label{sec-tests}

Dans cette étude, nous nous plaçons dans le cadre de la régression
linéaire avec des variables à une dimension. À chaque date \(t\), la
variable \(y_t\) (e.g., taux de croissance du PIB) est expliquée par une
combinaison linéaire d'une constante et de \(p\) variables explicatives,
\(x_{1,t},\dots,x_{p,t}\) (soldes d'opinion, indices de production
industrielle, indicatrices, etc.) : \[
y_t=\alpha_{0}+\alpha_{1} x_{1,t}+\dots+\alpha_{p} x_{p,t} +\varepsilon_t 
\] où \(\varepsilon_t\) représente le terme d'erreur. En notant
\({\bf X}_t=\begin{pmatrix}1 & x_{1,t} &\cdots & x_{p,t} \end{pmatrix}\)
et
\({\bf \alpha}=\transp{\begin{pmatrix}\alpha_0 & \alpha_1 &\cdots & \alpha_p \end{pmatrix}}\),
cela s'écrit matriciellement~:
\begin{equation}\phantomsection\label{eq-mod-fixe}{
y_t={\bf X_t} \bf\alpha +\varepsilon_t.
}\end{equation}

Dans le cadre de la régression linéaire, les coefficients \(\bf\alpha\)
sont supposés constants dans le temps et estimés en utilisant l'ensemble
des données. Cela suppose donc que la relation économique entre les
différentes variables est stable dans le temps. Même si cette hypothèse
est généralement vraie sur le court-terme, elle peut être invalidée sur
le long-terme du fait de changements structurels (mesures économiques,
crises, changement de nomenclature, etc.). L'objectif de cette étude est
d'étudier différents modèles permettant de relâcher cette hypothèse de
constance des coefficients. Le modèle général s'écrit donc : \[
y_t={\bf X_t} \bf\alpha_t  +\varepsilon_t.
\] Pour faciliter l'utilisation des modèles ici présentés, le package
\faIcon{r-project} \texttt{tvCoef} \autocite{tvcoef} a été développé.
Leur implémentation est illustrée à travers l'exemple de la prévision du
taux de croissance trimestriel du PIB, noté \(y_t\), à partir du climat
des affaires France publié par l'Insee\footnote{ Cette série est
  disponible à l'URL
  \url{https://www.insee.fr/fr/statistiques/serie/001565530}.}. Ces
séries sont disponibles sous \faIcon{r-project} dans la base de donnée
\texttt{tvCoef::gdp}\footnote{ Les données sont disponibles dans le
  package \texttt{tvCoef} ont été téléchargées le 15 mars 2024 et
  peuvent donc différer de celles actuellement disponibles.} :

\begin{itemize}
\item
  \texttt{growth\_gdp} correspond au taux de croissance trimestriel du
  PIB ;
\item
  \texttt{bc\_fr\_m1} correspond au climat des affaires au premier mois
  de chaque trimestre (la valeur de 2000T1 correspond à la valeur de
  janvier 2000, celle de 2000T2 à celle d'avril 2000, etc.) ;
\item
  \texttt{diff\_bc\_fr\_m1} correspond à la différenciation
  trimestrielle de la variable précédente (la valeur de 2000T1
  correspond à la différence du climat des affaires entre janvier 2000
  et octobre 1999).
\end{itemize}

Les graphiques de ces variables sont disponibles dans l'annexe
\ref{sec-an-graph}.

Le modèle s'écrit donc : \[
y_t=\alpha_0 + \alpha_1\times climat\_fr_t^{m_1} + \alpha_2\times \Delta climat\_fr_t^{m_1}+\varepsilon_t.
\] Il est estimé en utilisant les données entre les années 1980 et 2019.
Le climat des affaires France publié par l'Insee est un indicateur
mensuel normalisé de moyenne 100 et d'écart-type 10 sur l'ensemble de la
période de publication (de janvier 1977 à février 2024 dans notre cas).
Lorsqu'il est à sa moyenne, la croissance du PIB est de
\(100\times\alpha_1\). Afin de faciliter l'interprétation de la
constante, le climat des affaires est renormalisée à 0 et le modèle est
donc : \[
y_t=\alpha_0 + \alpha_1\times (climat\_fr_t^{m_1}-100) + \alpha_2\times \Delta climat\_fr_t^{m_1}+\varepsilon_t.
\] Sous \faIcon{r-project}, ce modèle peut être estimé en utilisant la
fonction \texttt{stats::lm()}. Toutefois, nous recommandons d'utiliser
le package \texttt{dynlm} \autocite{dynlm} qui offre une plus grande
flexibilité dans la définition des modèles et permet de conserver le
format série temporelle dans les fonctions de \texttt{tvCoef}.

\begin{Shaded}
\begin{Highlighting}[]
\FunctionTok{library}\NormalTok{(tvCoef)}
\FunctionTok{library}\NormalTok{(dynlm)}
\NormalTok{data\_gdp }\OtherTok{\textless{}{-}} \FunctionTok{window}\NormalTok{(gdp, }\AttributeTok{start =} \DecValTok{1980}\NormalTok{, }\AttributeTok{end =} \FunctionTok{c}\NormalTok{(}\DecValTok{2019}\NormalTok{, }\DecValTok{4}\NormalTok{))}
\CommentTok{\# Renormalisaiton à 0 du climat des affaires :}
\NormalTok{bc\_variables }\OtherTok{\textless{}{-}} \FunctionTok{c}\NormalTok{(}\StringTok{"bc\_fr\_m1"}\NormalTok{, }\StringTok{"bc\_fr\_m2"}\NormalTok{, }\StringTok{"bc\_fr\_m3"}\NormalTok{)}
\NormalTok{data\_gdp[, bc\_variables] }\OtherTok{\textless{}{-}}\NormalTok{ data\_gdp[, bc\_variables] }\SpecialCharTok{{-}} \DecValTok{100}
\NormalTok{reg\_lin }\OtherTok{\textless{}{-}} \FunctionTok{dynlm}\NormalTok{(}
  \AttributeTok{formula =}\NormalTok{ growth\_gdp }\SpecialCharTok{\textasciitilde{}}\NormalTok{ bc\_fr\_m1 }\SpecialCharTok{+}\NormalTok{ diff\_bc\_fr\_m1,}
  \AttributeTok{data =}\NormalTok{ data\_gdp}
\NormalTok{)}
\CommentTok{\# \# Equivalent à :}
\CommentTok{\# reg\_lin \textless{}{-} dynlm(}
\CommentTok{\#   formula = growth\_gdp \textasciitilde{} bc\_fr\_m1 + diff(bc\_fr\_m1, 1),}
\CommentTok{\#   \# Date de début changée car on perd une donnée avec la différenciation}
\CommentTok{\#   data = window(gdp, start = c(1979, 4), end = c(2019, 4))}
\CommentTok{\# )}
\FunctionTok{summary}\NormalTok{(reg\_lin)}
\end{Highlighting}
\end{Shaded}

\begin{verbatim}

Time series regression with "ts" data:
Start = 1980(1), End = 2019(4)

Call:
dynlm(formula = growth_gdp ~ bc_fr_m1 + diff_bc_fr_m1, data = data_gdp)

Residuals:
     Min       1Q   Median       3Q      Max 
-1.30140 -0.23883  0.02808  0.24487  0.94292 

Coefficients:
              Estimate Std. Error t value Pr(>|t|)    
(Intercept)   0.447207   0.030742  14.547  < 2e-16 ***
bc_fr_m1      0.020473   0.003171   6.456 1.28e-09 ***
diff_bc_fr_m1 0.044228   0.007412   5.967 1.55e-08 ***
---
Signif. codes:  0 '***' 0.001 '**' 0.01 '*' 0.05 '.' 0.1 ' ' 1

Residual standard error: 0.3888 on 157 degrees of freedom
Multiple R-squared:  0.3823,    Adjusted R-squared:  0.3744 
F-statistic: 48.58 on 2 and 157 DF,  p-value: < 2.2e-16
\end{verbatim}

Le modèle estimé est donc :

\[y_t=0,45 + 0,02\times (climat\_fr_t^{m_1} -100)+ 0,04\times \Delta climat\_fr_t^{m_1}+{\hat\varepsilon}_t,\]

\subsection{Test de rupture brutale}\label{sec-test-baiperron}

L'idée la plus simple pour tester s'il y a une rupture dans l'estimation
des coefficients à une date \(t_1\), est d'estimer deux sous-modèles
avant et après cette date : \[
\begin{cases}
\forall t \leq t_1 :\quad y_t = \alpha_0' + \alpha_1' climat\_fr_t + \alpha_2' \Delta climat\_fr_t + \varepsilon_t' \\
\forall t > t_1 :\quad y_t = \alpha_0'' + \alpha_1'' climat\_fr_t + \alpha_2'' \Delta climat\_fr_t + \varepsilon_t''
\end{cases}.
\] Il ne reste ensuite qu'à tester si les coefficients estimés entre les
deux sous-périodes sont égaux~: \(\alpha_0' = \alpha_0''\),
\(\alpha_1' = \alpha_1''\) et \(\alpha_2' = \alpha_2''.\) L'hypothèse
alternative est qu'au moins un des coefficients est différent entre les
deux sous-périodes. C'est le principe du test de \textcite{chowtest}.

L'inconvénient est que cela suppose d'avoir un \emph{a priori} sur la
date de la rupture à tester. Pour palier à ce problème,
\textcite{bai2003computation} ont proposé un algorithme efficace afin de
chercher la présence de ruptures multiples dans des modèles de
régression linéaire. Cet algorithme a été implémenté sous
\faIcon{r-project} dans le package \texttt{strucchange}
\autocite{strucchangeBP}. La fonction
\texttt{strucchange::breakpoints()} permet de chercher les ruptures et
la fonction \texttt{strucchange::breakdates()} permet d'extraire
facilement les dates associées. Le package \texttt{tvCoef} implémente
une méthode \texttt{breakpoints.lm()} afin de pouvoir directement
appliquer cette fonction aux régressions linéaires estimées :

\begin{Shaded}
\begin{Highlighting}[]
\FunctionTok{library}\NormalTok{(strucchange)}
\NormalTok{bp }\OtherTok{\textless{}{-}} \FunctionTok{breakpoints}\NormalTok{(reg\_lin)}
\FunctionTok{breakdates}\NormalTok{(bp)}
\end{Highlighting}
\end{Shaded}

\begin{verbatim}
[1] 2000.5
\end{verbatim}

Une seule rupture est détectée au 2000T3. Un intervalle de confiance
autour de la date détectée peut être calculé en utilisant la fonction
\texttt{stats::confint()} :

\begin{Shaded}
\begin{Highlighting}[]
\FunctionTok{breakdates}\NormalTok{(}\FunctionTok{confint}\NormalTok{(bp))}
\end{Highlighting}
\end{Shaded}

\begin{verbatim}
    2.5 % breakpoints 97.5 %
1 1995.75      2000.5   2005
\end{verbatim}

L'incertitude autour de la date détectée est grande ! Il y a 95 \% de
chance que la rupture soit comprise entre 1995T4 et 2005T1.

Cet algorithme est très simple à utiliser mais possède plusieurs
inconvénients :

\begin{itemize}
\item
  L'implémentation sous \faIcon{r-project} de l'algorithme de Bai et
  Perron ne permet pas de chercher des ruptures sur un sous-ensemble de
  variables : on ne cherche des ruptures que sur l'ensemble du modèle.
  Par exemple, on ne peut pas tester \(\alpha_2' = \alpha_2''\) dans le
  modèle : \[
  \begin{cases}
  \forall t \leq t_1 :\quad y_t = \alpha_0 + \alpha_1 climat\_fr_t + \alpha_2' \Delta climat\_fr_t + \varepsilon_t' \\
  \forall t > t_1 :\quad y_t = \alpha_0 + \alpha_1 climat\_fr_t + \alpha_2'' \Delta climat\_fr_t + \varepsilon_t''
  \end{cases}.
  \]
\item
  Il y a une instabilité sur le choix de la date et il suppose que la
  rupture est brutale à une certaine date. Si la rupture est brutale, le
  statisticien doit pouvoir expliquer son origine (changement de
  nomenclature, de champ dans les données, crise\ldots) et a déjà un
  \emph{a priori} sur la date de rupture. Si l'on n'a aucune information
  sur la présence d'une rupture, on peut raisonnablement penser que
  celle-ci n'est pas brutale mais que la relation entre les variables a
  évolué de manière progressive dans le temps.
\end{itemize}

\subsection{Test de constance des coefficients}\label{sec-hansen-test}

Alors que l'algorithme de Bai et Perron cherche une date spécifique où
il y aurait une rupture dans les modèles, \textcite{hansen1992testing}
propose une procédure permettant de tester uniquement si les
coefficients sont constants ou non sans hypothèse sur la forme de la
rupture (brutale ou non) et sur la date de la rupture.

La modélisation générale de la régression linéaire s'écrit :
\begin{align*}
y_t&=\alpha_{0}x_{0,t}+\alpha_{1} x_{1,t}+\dots+\alpha_{p} x_{p,t} +\varepsilon_t  \\
&= {\bf X_t} \bf\alpha  +\varepsilon_t\\
\E{\varepsilon_t|x_t}&=0 \text{ (exogénéité stricte)} \\
\E{\varepsilon_t^2}&=\sigma_t^2\text{ et } \underset{n\to\infty}{\lim}\frac{1}{n}\sum_{t=1}^n\sigma_t^2=\sigma.
\end{align*} On suppose également que toutes les variables sont
faiblement dépendantes\footnote{ Des variables sont faiblement
  dépendantes lorsque leur corrélation tend vers 0. Dans ce cas, le
  théorème central limite s'applique et les estimateurs sont
  asymptotiquement normaux (sans avoir besoin de supposer que les
  variables sont iid).} (cas général de la régression linéaire).

Les variables ne doivent donc pas contenir de tendance déterministe ou
stochastique (comme des racines unitaires).

Le test consiste à vérifier si l'ensemble des paramètres
\((\bf \alpha,\sigma^2)\) sont constants. L'hypothèse alternative est
qu'au moins un paramètre suit une martingale.

Notons \({\hat \varepsilon}_t =y_t- {\bf X_t} \hat{\bf\alpha}\) et \[
f_{i,t} = \begin{cases}
x_{i,t}\hat \varepsilon_t &\text{ si }i\leq p\\
\hat \varepsilon_t^2 - \hat \sigma^2&\text{ si }i=p+1
\end{cases}
\text{ et }S_{i,t} = \sum_{j=1}^tf_{i,j}.
\] D'après les conditions de premier ordre \(S_{i,n}=0.\)

Le test individuel de constance du coefficient du paramètre \(i\) est :
\[
L_i=\frac{1}{nV_i}\sum_{t=1}^nS_{i,t}^2\qquad
\text{avec }V_i=\sum_{t=1}^nf_{i,t}^2.
\]

Notons : \[
\bf f_t= \begin{pmatrix}
f_{1,t} \\ \vdots \\ f_{p+1,t}
\end{pmatrix} \text{ et }
\bf S_t= \begin{pmatrix}
S_{1,t} \\ \vdots \\ S_{p+1,t}
\end{pmatrix}.
\] Le test joint de constance de l'ensemble des paramètres est : \[
L_c = \frac{1}{n}
\sum_{t=1}^n\transp{\bf S_t}\bf V^{-1}\bf S_t
\text{ avec }\bf V=\sum_{t=1}^n\bf f_{t}\transp{\bf f_{t}}.
\] Il s'adapte facilement à un test joint de constance d'un
sous-ensemble de paramètres en utilisant des sous-vecteurs de
\(\bf f_t\) et \(\bf S_t.\) Toutefois, si le modèle contient des
indicatrices alors le test joint ne pourra pas être calculé\footnote{
  Dans ce cas, la matrice \(\bf V\) n'est pas inversible car la colonne
  associée à l'indicatrice sera proche de 0. Si la
  \(i\)\textsuperscript{e} variable est une indicatrice à la date
  \(t_0\), \(f_{i,t}=x_{i,t}\hat\varepsilon_t\) sera égal à 0 pour \(t\)
  différent de \(t_0\) (car \(x_{i,t}=0\)) et
  \(f_{i,t_{0}}=x_{i,t_{0}}\hat\varepsilon_{t_0}=\hat\varepsilon_{t_0}= 0\)
  car l'ajout d'une indicatrice conduit généralement les résidus à être
  nuls sur la date associée.}.

Sous l'hypothèse nulle de constance des paramètres, les \(S_{i,t}\)
devraient tendre vers 0 (à la manière d'une \emph{tied-down random
walk}, c'est-à-dire une marche aléatoire où l'on a contraint la première
observation à être égale à la dernière observation) : les statistiques
de test \(L_i\) et \(L_c\) devraient donc être petites. Sous l'hypothèse
alternative d'instabilité des paramètres, la somme cumulée des
\(S_{i,t}\) devrait ne pas être de moyenne nulle dans un sous-ensemble
de l'échantillon et la statistique de test devrait être élevée.
L'hypothèse nulle de stabilité des coefficients est donc rejetée lorsque
la statistique de test est grande. Sous l'hypothèse nulle, la loi de
distribution asymptotique est non standard, les valeurs critiques sont
présentées dans la table~\ref{tbl-hansen-table}.

\begin{longtable}[]{@{}rllllll@{}}

\caption{\label{tbl-hansen-table}Valeurs critiques asymptotiques pour
\(L_c\) en fonction du nombre de paramètres testés (1 degré de liberté
pour \(L_i\)).}

\tabularnewline

\toprule\noalign{}
Degrés de liberté & 1 \% & 2,5 \% & 5 \% & 7,5 \% & 10 \% & 20 \% \\
\midrule\noalign{}
\endhead
\bottomrule\noalign{}
\endlastfoot
1 & 0,748 & 0,593 & 0,470 & 0,398 & 0,353 & 0,243 \\
2 & 1,07 & 0,898 & 0,749 & 0,670 & 0,610 & 0,469 \\
3 & 1,35 & 1,16 & 1,01 & 0,913 & 0,846 & 0,679 \\
4 & 1,60 & 1,39 & 1,24 & 1,14 & 1,07 & 0,883 \\
5 & 1,88 & 1,63 & 1,47 & 1,36 & 1,28 & 1,08 \\
6 & 2,12 & 1,89 & 1,68 & 1,58 & 1,49 & 1,28 \\
7 & 2,35 & 2,10 & 1,90 & 1,78 & 1,69 & 1,46 \\
8 & 2,59 & 2,33 & 2,11 & 1,99 & 1,89 & 1,66 \\
9 & 2,82 & 2,55 & 2,32 & 2,19 & 2,10 & 1,85 \\
10 & 3,05 & 2,76 & 2,54 & 2,40 & 2,29 & 2,03 \\
11 & 3,27 & 2,99 & 2,75 & 2,60 & 2,49 & 2,22 \\
12 & 3,51 & 3,18 & 2,96 & 2,81 & 2,69 & 2,41 \\
13 & 3,69 & 3,39 & 3,15 & 3,00 & 2,89 & 2,59 \\
14 & 3,90 & 3,60 & 3,34 & 3,19 & 3,08 & 2,77 \\
15 & 4,07 & 3,81 & 3,54 & 3,38 & 3,26 & 2,95 \\
16 & 4,30 & 4,01 & 3,75 & 3,58 & 3,46 & 3,14 \\
17 & 4,51 & 4,21 & 3,95 & 3,77 & 3,64 & 3,32 \\
18 & 4,73 & 4,40 & 4,14 & 3,96 & 3,83 & 3,50 \\
19 & 4,92 & 4,60 & 4,33 & 4,16 & 4,03 & 3,69 \\
20 & 5,13 & 4,79 & 4,52 & 4,36 & 4,22 & 3,86 \\

\end{longtable}

{\footnotesize\raggedright 

\textbf{Source} : \textcite{hansen1990lagrange}. Table également
disponible avec la commande \texttt{tvCoef::hansen\_table}.

}

Ce test est implémenté dans la fonction \texttt{tvCoef::hansen\_test()}.
Par défaut, le test joint ne comprend pas le test de constance de la
variance (\texttt{sigma\ =\ FALSE}).

\begin{Shaded}
\begin{Highlighting}[]
\FunctionTok{hansen\_test}\NormalTok{(reg\_lin)}
\end{Highlighting}
\end{Shaded}

\begin{verbatim}
                   L Stat Reject at 5%
(Intercept)   1.7744 0.47         TRUE
bc_fr_m1      0.1529 0.47        FALSE
diff_bc_fr_m1 0.2052 0.47        FALSE
Variance      0.1240 0.47        FALSE
Joint Lc      2.0347 1.47         TRUE
\end{verbatim}

Sur notre modèle de prévision de la croissance, le test joint conclut à
la non constance des coefficients. Le test individuel sur le coefficient
associé au climat des affaires en différence conclut à sa constance (au
seuil de 5 \%). En revanche, le test de Hansen individuel conclut à la
non-constance des coefficients associés à la constante et au climat des
affaires en niveau au seuil de 5 \%. La non constance de ces deux
paramètres peut être vérifiée en utilisant un test joint sur ces deux
variables :

\begin{Shaded}
\begin{Highlighting}[]
\FunctionTok{hansen\_test}\NormalTok{(reg\_lin, }\AttributeTok{var =} \FunctionTok{c}\NormalTok{(}\DecValTok{1}\NormalTok{, }\DecValTok{2}\NormalTok{))}
\end{Highlighting}
\end{Shaded}

\begin{verbatim}
                   L Stat Reject at 5%
(Intercept)   1.7744 0.47         TRUE
bc_fr_m1      0.1529 0.47        FALSE
diff_bc_fr_m1 0.2052 0.47        FALSE
Variance      0.1240 0.47        FALSE
Joint Lc      1.9322 1.24         TRUE
\end{verbatim}

Le test de Hansen peut être vu comme une extension des tests de
stabilité CUSUM (\emph{cumulative sum control chart}) et CUSUM sur les
carrés (pour le test sur la variance). Il est robuste à
l'hétéroscédasticité. En appliquant les mêmes formules au modèle «
transformé », ce test est également robuste à la prise en compte de
l'autocorrélation via les moindres carrés généralisés. En revanche, il
suppose que toutes les variables sont stationnaires : il ne peut donc
directement s'appliquer sur des modèles du type modèle à correction
d'erreur. Dans ce cas, une loi asymptotique différente doit être
utilisée\footnote{ Voir par exemple \textcite{hansen1992I1}. Une
  implémentation sous \faIcon{r-project} de ce cas est disponible sous
  \url{https://users.ssc.wisc.edu/~bhansen/progs/jbes_92.html}.}. Si le
modèle est estimé en deux étapes par la méthode de
\textcite{engle1987co}, le test peut en revanche s'appliquer sur la
seconde estimation (estimation des paramètres de court-terme et de la
force de rappel).

\section{Description des méthodes}\label{sec-desc-meth}

Si un des tests précédents conclut à la non constance des coefficients
du modèle estimé c'est qu'il est mal spécifié et donc qu'il faut
utiliser une modélisation alternative qui pourrait notamment provenir
d'un problème de variables omises. Dans cette étude, nous supposons que
le problème de spécification provient des observations récentes et qu'il
n'est pas nécessaire de faire un ajout de nouvelles variables
explicatives pour le régler. Dans certains cas, par exemple pour prendre
en compte la crise du COVID-19, il peut être utile d'ajouter des
variables supplémentaires (e.g., des indicatrices).

Trois méthodes sont étudiées dans cette étude :

\begin{itemize}
\item
  la régression linéaire par morceaux (section~\ref{sec-reg-morceaux}) ;
\item
  la régression locale (section~\ref{sec-reg-locale}) ;
\item
  la régression avec coefficients stochastiques (estimés par une
  modélisation espace-état~; section~\ref{sec-ssm}).
\end{itemize}

\subsection{Régression par morceaux}\label{sec-reg-morceaux}

La régression par morceaux est la modélisation la plus simple : elle
consiste à estimer le modèle sur un sous-ensemble des données. La
modélisation est similaire à celle de la procédure de Bai et Perron
puisque cette dernière donne directement les « morceaux » : entre les
dates de ruptures.

Par exemple, pour le modèle de prévision de la croissance, deux
régressions seraient estimées en utilisant les données avant et après
2000T3.

Deux méthodes d'estimations sont possibles :

\begin{enumerate}
\def\labelenumi{\arabic{enumi}.}
\item
  Une régression en une étape est faite en dédoublant les régresseurs en
  fonction de la date de rupture (fonction
  \texttt{tvCoef::piece\_reg()}) : \begin{align*}
  y_t &= \alpha_0\1_{t\leq 2000T3} + \alpha_1 climat\_fr_t\1_{t\leq 2000T3} + \alpha_2 \Delta climat\_fr_t\1_{t\leq 2000T3} + \\
  &\phantom{=} \alpha_0'\1_{t > 2000T3} + \alpha_1' climat\_fr_t\1_{t > 2000T3} + \alpha_2' \Delta climat\_fr_t\1_{t > 2000T3} + \varepsilon_t
  \end{align*}
\item
  En effectuant deux régressions linéaires distinctes (fonction
  \texttt{tvCoef::bp\_lm()}) : \[
  \begin{cases}
  \forall t \leq 2000T3 :\quad y_t = \alpha_0 + \alpha_1 climat\_fr_t + \alpha_2 \Delta climat\_fr_t + \varepsilon_t \\
  \forall t > 2000T3 :\quad y_t = \alpha_0' + \alpha_1' climat\_fr_t + \alpha_2' \Delta climat\_fr_t + \varepsilon_t'
  \end{cases}.
  \]
\end{enumerate}

Dans les deux cas les coefficients estimés sont les mêmes mais les
écarts-types seront en général différents. En effet, dans la première
modélisation on suppose que la variance du résidu est constante sur
l'ensemble de la période alors que dans la seconde la variance est
supposée constante sur les deux sous-périodes. Par défaut les fonctions
\texttt{tvCoef::piece\_reg()} et \texttt{tvCoef::bp\_lm()} calculent les
dates de ruptures en appliquant la fonction
\texttt{strucchange::breakdates()} sur le modèle de régression linéaire
en paramètre. Cette date de rupture peut toutefois être manuellement
spécifiée en utilisant le paramètre \texttt{break\_date}.

\begin{Shaded}
\begin{Highlighting}[]
\NormalTok{reg\_morc }\OtherTok{\textless{}{-}} \FunctionTok{piece\_reg}\NormalTok{(reg\_lin)}
\NormalTok{bp\_lm }\OtherTok{\textless{}{-}} \FunctionTok{bp\_lm}\NormalTok{(reg\_lin)}
\FunctionTok{coef}\NormalTok{(reg\_morc}\SpecialCharTok{$}\NormalTok{model)}
\end{Highlighting}
\end{Shaded}

\begin{verbatim}
 `(Intercept)_2000.5`       bc_fr_m1_2000.5  diff_bc_fr_m1_2000.5 
           0.57683470            0.02233539            0.03290511 
`(Intercept)_2019.75`      bc_fr_m1_2019.75 diff_bc_fr_m1_2019.75 
           0.31104591            0.02013180            0.05475011 
\end{verbatim}

\begin{Shaded}
\begin{Highlighting}[]
\FunctionTok{c}\NormalTok{(}\FunctionTok{coef}\NormalTok{(bp\_lm}\SpecialCharTok{$}\NormalTok{model[[}\DecValTok{1}\NormalTok{]]), }\FunctionTok{coef}\NormalTok{(bp\_lm}\SpecialCharTok{$}\NormalTok{model[[}\DecValTok{2}\NormalTok{]]))}
\end{Highlighting}
\end{Shaded}

\begin{verbatim}
  (Intercept)      bc_fr_m1 diff_bc_fr_m1   (Intercept)      bc_fr_m1 
   0.57683470    0.02233539    0.03290511    0.31104591    0.02013180 
diff_bc_fr_m1 
   0.05475011 
\end{verbatim}

Dans la majorité des cas, la variance des erreurs n'a pas de raison
d'être différente selon la période considérée, et nous suggérons de
privilégier la première modélisation car elle offre plus de flexibilité,
notamment pour fixer les coefficients de certaines variables.

Dans notre exemple, le test d'Hansen concluait à la constance du
coefficient du climat des affaires en différence~: les coefficients
estimés avant et après la rupture devraient donc être égaux. Cette
égalité peut être testée en utilisant un test de Fisher, par exemple
avec la fonction \texttt{car::linearHypothesis()} \autocite{car}. Ici on
rejette l'hypothèse nulle d'égalité de tous les coefficients avant et
après la rupture : la prise en compte de la rupture est donc justifiée.
On ne rejette pas l'hypothèse nulle d'égalité du coefficient du climat
des affaires en niveau seulement, le modèle pourrait donc être
simplifié.

\begin{Shaded}
\begin{Highlighting}[]
\CommentTok{\# On rejette l\textquotesingle{}hypothèse nulle de constance de tous les coefficients}
\NormalTok{car}\SpecialCharTok{::}\FunctionTok{linearHypothesis}\NormalTok{(}
\NormalTok{  reg\_morc}\SpecialCharTok{$}\NormalTok{model,}
  \FunctionTok{c}\NormalTok{(}
    \StringTok{"\textasciigrave{}(Intercept)\_2000.5\textasciigrave{} = \textasciigrave{}(Intercept)\_2019.75\textasciigrave{}"}\NormalTok{,}
    \StringTok{"bc\_fr\_m1\_2000.5 = bc\_fr\_m1\_2019.75"}\NormalTok{,}
    \StringTok{"diff\_bc\_fr\_m1\_2000.5 = diff\_bc\_fr\_m1\_2019.75"}
\NormalTok{    )}
\NormalTok{  ,}
  \AttributeTok{test =} \StringTok{"F"}\NormalTok{)}
\end{Highlighting}
\end{Shaded}

\begin{verbatim}
Linear hypothesis test

Hypothesis:
(Intercept)_2000.5` - Intercept)_2019.75` = 0
bc_fr_m1_2000.5 - bc_fr_m1_2019.75 = 0
diff_bc_fr_m1_2000.5 - diff_bc_fr_m1_2019.75 = 0

Model 1: restricted model
Model 2: y ~ 0 + (`(Intercept)_2000.5` + bc_fr_m1_2000.5 + diff_bc_fr_m1_2000.5 + 
    `(Intercept)_2019.75` + bc_fr_m1_2019.75 + diff_bc_fr_m1_2019.75)

  Res.Df    RSS Df Sum of Sq      F    Pr(>F)    
1    157 23.733                                  
2    154 20.579  3    3.1549 7.8699 6.411e-05 ***
---
Signif. codes:  0 '***' 0.001 '**' 0.01 '*' 0.05 '.' 0.1 ' ' 1
\end{verbatim}

\begin{Shaded}
\begin{Highlighting}[]
\CommentTok{\# On ne rejette pas (H0) constance du climat des affaires en différence}
\NormalTok{car}\SpecialCharTok{::}\FunctionTok{linearHypothesis}\NormalTok{(}
\NormalTok{  reg\_morc}\SpecialCharTok{$}\NormalTok{model,}
  \FunctionTok{c}\NormalTok{(}
    \StringTok{"diff\_bc\_fr\_m1\_2000.5 = diff\_bc\_fr\_m1\_2019.75"}
\NormalTok{    ),}
  \AttributeTok{test =} \StringTok{"F"}\NormalTok{)}
\end{Highlighting}
\end{Shaded}

\begin{verbatim}
Linear hypothesis test

Hypothesis:
diff_bc_fr_m1_2000.5 - diff_bc_fr_m1_2019.75 = 0

Model 1: restricted model
Model 2: y ~ 0 + (`(Intercept)_2000.5` + bc_fr_m1_2000.5 + diff_bc_fr_m1_2000.5 + 
    `(Intercept)_2019.75` + bc_fr_m1_2019.75 + diff_bc_fr_m1_2019.75)

  Res.Df    RSS Df Sum of Sq      F Pr(>F)
1    155 20.905                           
2    154 20.579  1   0.32652 2.4435 0.1201
\end{verbatim}

\begin{Shaded}
\begin{Highlighting}[]
\CommentTok{\# Pour les autres variables, faire attention à l\textquotesingle{}interprétation des tests }
\CommentTok{\# individuels : on ne rejette pas (H0) constance du climat des affaires}
\CommentTok{\# en niveau}
\NormalTok{car}\SpecialCharTok{::}\FunctionTok{linearHypothesis}\NormalTok{(}
\NormalTok{  reg\_morc}\SpecialCharTok{$}\NormalTok{model,}
  \FunctionTok{c}\NormalTok{(}
    \StringTok{"bc\_fr\_m1\_2000.5 = bc\_fr\_m1\_2019.75"}
\NormalTok{    ),}
  \AttributeTok{test =} \StringTok{"F"}\NormalTok{)}
\end{Highlighting}
\end{Shaded}

\begin{verbatim}
Linear hypothesis test

Hypothesis:
bc_fr_m1_2000.5 - bc_fr_m1_2019.75 = 0

Model 1: restricted model
Model 2: y ~ 0 + (`(Intercept)_2000.5` + bc_fr_m1_2000.5 + diff_bc_fr_m1_2000.5 + 
    `(Intercept)_2019.75` + bc_fr_m1_2019.75 + diff_bc_fr_m1_2019.75)

  Res.Df    RSS Df Sum of Sq      F Pr(>F)
1    155 20.596                           
2    154 20.579  1  0.017314 0.1296 0.7194
\end{verbatim}

\begin{Shaded}
\begin{Highlighting}[]
\CommentTok{\# On ne rejette pas (H0) constance de la constante}
\NormalTok{car}\SpecialCharTok{::}\FunctionTok{linearHypothesis}\NormalTok{(}
\NormalTok{  reg\_morc}\SpecialCharTok{$}\NormalTok{model,}
  \FunctionTok{c}\NormalTok{(}
    \StringTok{"\textasciigrave{}(Intercept)\_2000.5\textasciigrave{} = \textasciigrave{}(Intercept)\_2019.75\textasciigrave{}"}
\NormalTok{    ),}
  \AttributeTok{test =} \StringTok{"F"}\NormalTok{)}
\end{Highlighting}
\end{Shaded}

\begin{verbatim}
Linear hypothesis test

Hypothesis:
(Intercept)_2000.5` - Intercept)_2019.75` = 0

Model 1: restricted model
Model 2: y ~ 0 + (`(Intercept)_2000.5` + bc_fr_m1_2000.5 + diff_bc_fr_m1_2000.5 + 
    `(Intercept)_2019.75` + bc_fr_m1_2019.75 + diff_bc_fr_m1_2019.75)

  Res.Df    RSS Df Sum of Sq      F   Pr(>F)    
1    155 23.387                                 
2    154 20.579  1    2.8087 21.019 9.35e-06 ***
---
Signif. codes:  0 '***' 0.001 '**' 0.01 '*' 0.05 '.' 0.1 ' ' 1
\end{verbatim}

\begin{Shaded}
\begin{Highlighting}[]
\CommentTok{\# On rejette (H0) constance de la constante + climat des affaires en niveau}
\NormalTok{car}\SpecialCharTok{::}\FunctionTok{linearHypothesis}\NormalTok{(}
\NormalTok{  reg\_morc}\SpecialCharTok{$}\NormalTok{model,}
  \FunctionTok{c}\NormalTok{(}
    \StringTok{"\textasciigrave{}(Intercept)\_2000.5\textasciigrave{} = \textasciigrave{}(Intercept)\_2019.75\textasciigrave{}"}\NormalTok{,}
    \StringTok{"bc\_fr\_m1\_2000.5 = bc\_fr\_m1\_2019.75"}
\NormalTok{  ),}
  \AttributeTok{test =} \StringTok{"F"}\NormalTok{)}
\end{Highlighting}
\end{Shaded}

\begin{verbatim}
Linear hypothesis test

Hypothesis:
(Intercept)_2000.5` - Intercept)_2019.75` = 0
bc_fr_m1_2000.5 - bc_fr_m1_2019.75 = 0

Model 1: restricted model
Model 2: y ~ 0 + (`(Intercept)_2000.5` + bc_fr_m1_2000.5 + diff_bc_fr_m1_2000.5 + 
    `(Intercept)_2019.75` + bc_fr_m1_2019.75 + diff_bc_fr_m1_2019.75)

  Res.Df    RSS Df Sum of Sq      F    Pr(>F)    
1    156 23.405                                  
2    154 20.579  2    2.8268 10.577 4.963e-05 ***
---
Signif. codes:  0 '***' 0.001 '**' 0.01 '*' 0.05 '.' 0.1 ' ' 1
\end{verbatim}

\begin{Shaded}
\begin{Highlighting}[]
\CommentTok{\# Pour fixer le coefficient associé au climat des affaires en niveau}
\NormalTok{reg\_morc2 }\OtherTok{\textless{}{-}} \FunctionTok{piece\_reg}\NormalTok{(reg\_lin, }\AttributeTok{fixed\_var =} \DecValTok{2}\NormalTok{)}
\FunctionTok{coef}\NormalTok{(reg\_morc2}\SpecialCharTok{$}\NormalTok{model)}
\end{Highlighting}
\end{Shaded}

\begin{verbatim}
 `(Intercept)_2000.5` `(Intercept)_2019.75`              bc_fr_m1 
           0.57620106            0.31037798            0.02147121 
 diff_bc_fr_m1_2000.5 diff_bc_fr_m1_2019.75 
           0.03337132            0.05419116 
\end{verbatim}

La qualité prédictive du nouveau modèle peut s'apprécier de plusieurs
façons, la plus classique étant la minimisation du critère d'information
d'Akaike (AIC et fonction \texttt{AIC()}) ou la minimisation des erreurs
de prévision hors échantillon (également appelées pseudo temps-réel,
fonction \texttt{tvCoef::oos\_prev()}). Pour le calcul des erreurs de
prévision hors échantillon, la méthodologie retenue consiste à calculer
pour chaque date \(t\) la prévision obtenue à la date \(t+1\) en
estimant le modèle à partir des observations disponibles jusqu'à la date
\(t\) uniquement. Avec cette méthode, appelée le \emph{leave-one-out
cross-validation}, on ne s'intéresse donc qu'à la qualité de prévision à
l'horizon d'un trimestre (ce qui est le cas d'utilisation pour les
modèles étudiés). Par ailleurs, minimiser l'AIC est asymptotiquement
équivalent à minimiser ces erreurs de prévision hors échantillon
\autocite{AIC}.

Sur notre exemple, et par rapport à la régression supposant des
coefficients constants dans le temps, la régression linéaire par
morceaux permet de minimiser ces deux critères~:

\begin{Shaded}
\begin{Highlighting}[]
\CommentTok{\# AIC minimisé :}
\FunctionTok{AIC}\NormalTok{(reg\_morc}\SpecialCharTok{$}\NormalTok{model) }\SpecialCharTok{\textless{}} \FunctionTok{AIC}\NormalTok{(reg\_lin)}
\end{Highlighting}
\end{Shaded}

\begin{verbatim}
[1] TRUE
\end{verbatim}

\begin{Shaded}
\begin{Highlighting}[]
\NormalTok{oos\_reg\_morc }\OtherTok{\textless{}{-}} \FunctionTok{oos\_prev}\NormalTok{(reg\_morc)}
\NormalTok{oos\_lm }\OtherTok{\textless{}{-}} \FunctionTok{oos\_prev}\NormalTok{(reg\_lin)}
\NormalTok{res\_oos }\OtherTok{\textless{}{-}} \FunctionTok{ts.union}\NormalTok{(oos\_lm}\SpecialCharTok{$}\NormalTok{residuals, oos\_reg\_morc}\SpecialCharTok{$}\NormalTok{residuals)}
\CommentTok{\# Les deux modèles étant équivalents avant la rupture,}
\CommentTok{\# on n\textquotesingle{}étudie les prévisions qu\textquotesingle{}après celle{-}ci}
\NormalTok{res\_oos }\OtherTok{\textless{}{-}} \FunctionTok{window}\NormalTok{(res\_oos, }\AttributeTok{start =} \DecValTok{2003}\NormalTok{)  }
\CommentTok{\# erreurs de prévision hors échantillon minimisées}
\FunctionTok{apply}\NormalTok{(res\_oos, }\DecValTok{2}\NormalTok{, rmse)}
\end{Highlighting}
\end{Shaded}

\begin{verbatim}
      oos_lm$residuals oos_reg_morc$residuals 
             0.4310894              0.4045755 
\end{verbatim}

La figure~\ref{fig-prev-piecereg} montre les erreurs de prévision hors
échantillon des deux modèles étudiés. Autour de la date de rupture, la
régression linéaire par morceaux produit des prévisions peu réalistes
(ce qui conduit à des erreurs élevées) : cela s'explique par le fait que
très peu d'observations sont utilisées pour estimer les coefficients
associés aux régresseurs après la rupture, les estimateurs sont donc peu
précis (grande variance). Par ailleurs, lors d'un vrai exercice en temps
réel, la date de rupture ne sera généralement connue et prise en compte
que plusieurs trimestres après celle-ci : l'erreur est alors réduite.
Pour les analyses automatiques hors échantillon, il faut donc faire
attention aux valeurs prédites autour de la rupture !

\begin{figure}

\caption{\label{fig-prev-piecereg}Erreurs de prévision de la croissance
du PIB à partir d'un modèle de régression linéaire et d'un modèle de
régression linéaire par morceaux.}

\centering{

\includegraphics{DT-tvcoef_files/figure-pdf/ffig-prev-piecereg-1.pdf}

\footnotesize\raggedright

\textbf{Source} : Insee (PIB et climat des affaires France entre 1980 et
2019 téléchargés le 15 mars 2024), calculs de l'auteur.

}

\end{figure}%

Comme indiqué dans la section~\ref{sec-test-baiperron}, l'inconvénient
de cette méthode provient du choix de la date de rupture lorsque
celle-ci n'est pas imposée par l'utilisateur. La
figure~\ref{fig-temps-reel-bp} montre la date de la rupture détectée par
la procédure de Bai et Perron en fonction de la date de fin d'estimation
du modèle de régression linéaire : aucune rupture n'est détectée avant
2009 ou lorsque le modèle est estimé en utilisant des données jusqu'en
2012T3-2016T1. En fonction de la date de fin d'estimation, la rupture
détectée automatiquement peut tout aussi bien être en 2000 qu'en 2001,
2004 ou 2006. Même s'il est possible que cela n'ait que très peu d'effet
sur les prévisions estimées en fin de période, l'interprétation faite du
modèle sera vraisemblablement différente~!

\begin{figure}

\caption{\label{fig-temps-reel-bp}Date de rupture détectée par
l'algorithme de Bai et Perron en fonction de la date de fin d'estimation
du modèle.}

\centering{

\includegraphics{DT-tvcoef_files/figure-pdf/ffig-temps-reel-bp-1.pdf}

\footnotesize\raggedright

\textbf{Source} : Insee (PIB et climat des affaires France entre 1980 et
2019 téléchargés le 15 mars 2024), calculs de l'auteur.

}

\end{figure}%

\subsection{De la régression mobile à la régression
locale}\label{sec-reg-locale}

La régression mobile est une des méthodes empiriques les plus simples
pour savoir si les coefficients évoluent dans le temps. Celle-ci
consiste à estimer des régressions sur des fenêtres glissantes et à
observer la courbe des coefficients estimés. En reprenant notre exemple
où les données commencent en 1980, avec une fenêtre fixe de 10 ans (par
exemple), cela consiste à estimer une première régression entre 1980T1
et 1989T4, une deuxième entre 1980T2 et 1990T1\ldots{} et une dernière
entre 2010T1 et 2019T4. Sous R cela peut par exemple s'estimer en
utilisant la fonction \texttt{roll::roll\_lm()} \autocite{roll} :

\begin{Shaded}
\begin{Highlighting}[]
\NormalTok{roll\_lm }\OtherTok{\textless{}{-}}\NormalTok{ roll}\SpecialCharTok{::}\FunctionTok{roll\_lm}\NormalTok{(}
  \AttributeTok{x =}\NormalTok{ data\_gdp[, }\FunctionTok{c}\NormalTok{(}\StringTok{"bc\_fr\_m1"}\NormalTok{, }\StringTok{"diff\_bc\_fr\_m1"}\NormalTok{)],}
  \AttributeTok{y =}\NormalTok{ data\_gdp[, }\StringTok{"growth\_gdp"}\NormalTok{],}
  \AttributeTok{width =} \DecValTok{4} \SpecialCharTok{*} \DecValTok{10}
\NormalTok{)}
\NormalTok{coef\_roll\_lm }\OtherTok{\textless{}{-}} \FunctionTok{ts}\NormalTok{(roll\_lm}\SpecialCharTok{$}\NormalTok{coefficients, }\AttributeTok{start =} \DecValTok{1980}\NormalTok{, }\AttributeTok{frequency =} \DecValTok{4}\NormalTok{)}
\end{Highlighting}
\end{Shaded}

La figure~\ref{fig-coef-rollreg} montre les coefficients estimés par
cette régression mobile. Seuls ceux estimés sur le climat des affaires
en différence montrent une rupture nette. Elle s'observe à partir de
2005, lorsque plus de la moitié des points de la fenêtre (5 ans) sont
estimés après la date de rupture détectée (2000T3).

\begin{figure}

\caption{\label{fig-coef-rollreg}Coefficients estimés par régression
linéaire, régression par morceaux et régression mobile.}

\centering{

\includegraphics{DT-tvcoef_files/figure-pdf/ffig-coef-rollreg-1.pdf}

\footnotesize\raggedright

\textbf{Lecture} : la régression mobile est estimée sur une fenêtre de
10 ans. Les coefficients estimés en 2000T1 correspondent aux
coefficients estimés entre 1980T2 et 2000T1.\\
\textbf{Note} : les échelles sont différentes entre les différents
graphiques.\\
\textbf{Source} : Insee (PIB et climat des affaires France entre 1980 et
2019 téléchargés le 15 mars 2024), calculs de l'auteur.

}

\end{figure}%

La régression mobile a l'avantage d'être très simple mais repose sur
plusieurs paramètres qui ont ici été fixés arbitrairement dont notamment
:

\begin{itemize}
\item
  La longueur de la fenêtre : elle doit être suffisamment large pour
  avoir des bonnes estimations mais suffisamment courte afin de
  permettre de prendre en compte les ruptures.
\item
  La date à laquelle les coefficients sont associés. Dans la fonction
  \texttt{roll::roll\_lm()} ils sont associés à la dernière date de la
  fenêtre : les coefficients de la date \(t\) correspondent à ceux
  obtenus en utilisant les données jusqu'à la date \(t.\) Ils auraient
  également pu être associés à la première date de la fenêtre ou encore
  à son milieu (coefficients de la date \(t\) estimés en utilisant
  autant d'observations avant et après \(t\)). Dans tous les cas une
  stratégie doit être adoptée afin de gérer les observations manquantes
  (dans notre exemple il s'agit donc d'estimer les coefficients avant
  1989).
\end{itemize}

La régression locale permet, grâce à une modélisation plus poussée, de
donner des solutions à ce problème. Dans ce papier nous détaillons la
modélisation utilisée dans la fonction \texttt{tvReg::tvLM()} développée
par \textcite{tvReg}\footnote{ D'autres packages sont disponibles pour
  effectuer une régression locale, dont par exemple \texttt{locfit} de
  \textcite{locfit}. Toutefois, nous avons ici privilégié le package
  \texttt{tvReg} du fait de sa simplicité d'utilisation et parce qu'il
  implémente également une fonction \texttt{tvReg::tvAR()} qui permet de
  prendre en compte de manière optimale les retards de la variable
  endogène (cas non étudié dans cette étude).}. On suppose ici que les
coefficients \(\bf\alpha_t\) de l'équation~\ref{eq-mod-fixe} dépendent
d'une variable aléatoire \(z_t\) : \(\bf\alpha_t=\alpha(z_t).\) Par
défaut \(z_t=t/T\) avec \(T\) le nombre d'observations~: les
coefficients dépendent donc d'une mesure normalisée du temps. On suppose
que la fonction \(\alpha\) est localement constante
(\(\alpha(z_t)\simeq \alpha(z)\), option par défaut) ou localement
linéaire (\(\alpha(z_t)\simeq \alpha(z)+\alpha'(z)(z_t-z)\)),
c'est-à-dire que pour toute date \(t\) on a pour toute date \(i\) proche
de \(t\) : \(\alpha(z_i)\simeq\alpha(z_t)\) ou
\(\alpha(z_i)\simeq\alpha(z_t)+\alpha'(z_t)(z_i-z_t).\) Cette
approximation locale est justifiée par le théorème de Taylor.

Pour chaque date \(t\), le coefficient \(\alpha_t=\alpha(z_t)\) est
obtenu par moindres carrés pondérés. Lorsque \(\alpha\) est supposé
localement constant il s'agit du système : \[
\hat{\bf\alpha_t}=\hat{\alpha}(z_t)=\underset{\bf\theta_0}\argmin\sum_{i=1}^T\left[y_i-{\bf X_i}\bf\theta_0 \right]^2K_{b_t}(z_i-z_t).
\] Lorsque \(\alpha\) est supposé localement linéaire il s'agit du
système : \[
(\hat{\alpha}(z_t), \hat{\alpha}'(z_t))=\underset{\bf \theta_0,\bf \theta_1}\argmin\sum_{i=1}^T\left[y_i-{\bf X_i}\bf\theta_0 - (z_i-z_t){\bf X_i}\bf\theta_1\right]^2K_{b_t}(z_i-z_t).
\] Avec
\(K_{b_t}(z_i-z_t)=\frac{1}{b_t}K\left(\frac{z_i-z_t}{b_t}\right)\) et
\(K(\cdot)\) une fonction de noyau. La fonction \(K\) permet de pondérer
les observations : pour l'estimation du coefficient à la date \(t\) on
accorde généralement plus d'importance (i.e., un poids plus important)
aux observations qui sont proches de \(t\) qu'à celles qui sont
éloignées de \(t.\) C'est une fonction positive, paire et intégrable
telle que \(\int_{-\infty}^{+\infty}K(u) \ud u=1.\) Trois noyaux sont
disponibles dans la fonction \texttt{tvReg::tvLM()} :

\begin{itemize}
\item
  Le cubique (\emph{triweight}, utilisé par défaut) : \[
  K(u)=\frac{35}{32}\left(
  1-
  \left\lvert
  u
  \right\lvert^2
  \right)^3\1_{[-1,1]}(u).
  \]
\item
  Le noyau d'Epanechnikov (ou parabolique) : \[
  K(u)=\frac{3}{4}\left(
  1-
  \left\lvert
  u
  \right\lvert^2
  \right)\1_{[-1,1]}(u).
  \]
\item
  Le noyau Gaussien : \[
  K(u)=\frac{1}{\sqrt{2\pi}}\exp\left(-\frac{1}{2}u^2\right).
  \]
\end{itemize}

Le paramètre \(b_t\) permet de calibrer la largeur de la fenêtre (i.e.,
le nombre de points utilisés pour chaque estimation). Il est
généralement supposé constant (\(b_t=b\)).

Dans notre exemple de prévision du PIB, \(T=160\) observations sont
utilisées. Avec \(z_t=t/T\) et indexant chaque observation entre 1 et
\(T,\) la régression mobile sur 15 ans où l'on affecte le coefficient de
la date \(t\) au milieu de la fenêtre d'estimation est donc retrouvée en
utilisant le noyau uniforme \(K(u)=\1_{[-1,1]}(x)\) avec
\(b_t=b=\frac{30}{160}.\) En effet, dans ce cas \(K(z_t-z_i)\ne0\) si et
seulement si \(|t-i|\leq30\) : on utilise donc 30 observations (soit
\(7,5\) ans) de chaque côté de \(t\) pour estimer le coefficient à la
date \(t.\)

Dans \texttt{tvReg}, le paramètre \(b\) est par défaut obtenu en
minimisant une statistique de validation croisée dans l'intervalle
\(\left[\frac{5}{T},20\right].\) Lorsque la valeur par défaut de \(b\)
est plus grande que 1, toutes les observations sont utilisées pour
l'estimation de chaque coefficient \(\bf\alpha_t.\) Plus \(b\) se
rapproche de 1 plus on se rapproche du cas de la régression linéaire
puisque dans ce cas les poids donnés par \(K\) tendent à être constants
pour toutes les observations. En effet, dans ce cas, pour \(T=160,\)
\(\frac{\max_u K(u)}{\min_u K(u)}\) est compris entre \(1,001\) et
\(1,008\) pour les noyaux cubiques, paraboliques et gaussiens.

Reprenons notre exemple de prévision du PIB avec une détection
automatique de la fenêtre.

\begin{Shaded}
\begin{Highlighting}[]
\NormalTok{reg\_loc }\OtherTok{\textless{}{-}}\NormalTok{ tvReg}\SpecialCharTok{::}\FunctionTok{tvLM}\NormalTok{(}
  \AttributeTok{formula =}\NormalTok{ growth\_gdp }\SpecialCharTok{\textasciitilde{}}\NormalTok{ bc\_fr\_m1 }\SpecialCharTok{+}\NormalTok{ diff\_bc\_fr\_m1,}
  \AttributeTok{data =}\NormalTok{ data\_gdp}
\NormalTok{)}
\end{Highlighting}
\end{Shaded}

\begin{verbatim}
Calculating regression bandwidth... bw =  0.7481989 
\end{verbatim}

\begin{Shaded}
\begin{Highlighting}[]
\FunctionTok{summary}\NormalTok{(reg\_loc)}
\end{Highlighting}
\end{Shaded}

\begin{verbatim}

Call: 
tvReg::tvLM(formula = growth_gdp ~ bc_fr_m1 + diff_bc_fr_m1, 
    data = data_gdp)

Class:  tvlm 

Summary of time-varying estimated coefficients: 
================================================ 
        (Intercept) bc_fr_m1 diff_bc_fr_m1
Min.         0.3100  0.02022       0.03454
1st Qu.      0.3711  0.02122       0.04146
Median       0.4548  0.02202       0.04795
Mean         0.4494  0.02194       0.04589
3rd Qu.      0.5306  0.02274       0.05091
Max.         0.5674  0.02303       0.05146

Bandwidth:  0.7482
Pseudo R-squared:  0.4499 
\end{verbatim}

La fenêtre estimée par défaut est de 0,75, c'est-à-dire que pour estimer
le coefficient à la date \(t\) on utilise au plus 30 ans avant et après
\(t\) : on utilise tous les points dans la majorité des cas. Cela
explique le caractère très lisse des coefficients
(figure~\ref{fig-coef-reg-mobile}). Bien qu'à n'importe quelle date
entre 1990 et 2010, tous les points sont utilisés pour estimer le
coefficient correspondant, les poids associés à ces points varient d'une
date à l'autre, si bien que le coefficient estimé n'est pas constant sur
cette période. Avec ce paramètre pour la fenêtre, les ruptures brutales
sont donc difficiles à prendre en compte.

\begin{figure}

\caption{\label{fig-coef-reg-mobile}Coefficients estimés par régression
linéaire, régression par morceaux et régression locale (avec
\(b=0,75\)).}

\centering{

\includegraphics{DT-tvcoef_files/figure-pdf/ffig-coef-reg-mobile-1.pdf}

\footnotesize\raggedright

\textbf{Note} : les échelles sont différentes entre les différents
graphiques.\\
\textbf{Source} : Insee (PIB et climat des affaires France entre 1980 et
2019 téléchargés le 15 mars 2024), calculs de l'auteur.

}

\end{figure}%

Un des inconvénients de la méthode de sélection automatique de la
fenêtre est que le critère utilisé (statistique de validation croisée)
est peu discriminant \autocite[voir notamment][]{Loader1999} : ce
critère peut prendre des valeurs très proches pour différentes valeurs
de la fenêtre alors que celle-ci a un impact fort sur l'interprétation
du modèle ! Cela a également pour effet que la méthode est peu stable
dans le temps (figure~\ref{fig-oos-bw}), ce qui augmente les sources de
révisions des simulations hors échantillon, calculables en utilisant la
fonction \texttt{oos\_prev()} :

\begin{Shaded}
\begin{Highlighting}[]
\NormalTok{oos\_reg\_loc }\OtherTok{\textless{}{-}} \FunctionTok{oos\_prev}\NormalTok{(reg\_loc)}
\NormalTok{oos\_bw }\OtherTok{\textless{}{-}} \FunctionTok{ts}\NormalTok{(}\FunctionTok{sapply}\NormalTok{(oos\_reg\_loc}\SpecialCharTok{$}\NormalTok{model, }\StringTok{\textasciigrave{}}\AttributeTok{[[}\StringTok{\textasciigrave{}}\NormalTok{,}\StringTok{"bw"}\NormalTok{),}
             \AttributeTok{end =} \FunctionTok{c}\NormalTok{(}\DecValTok{2019}\NormalTok{, }\DecValTok{4}\NormalTok{),}
             \AttributeTok{frequency =} \DecValTok{4}\NormalTok{)}
\end{Highlighting}
\end{Shaded}

\begin{figure}

\caption{\label{fig-oos-bw}Fenêtre \(b\) détectée automatiquement en
fonction de la date de fin d'estimation du modèle.}

\centering{

\includegraphics{DT-tvcoef_files/figure-pdf/ffig-oos-bw-1.pdf}

\footnotesize\raggedright

\textbf{Source} : Insee (PIB et climat des affaires France entre 1980 et
2019 téléchargés le 15 mars 2024), calculs de l'auteur.

}

\end{figure}%

L'estimation en temps réel revient à utiliser une fonction de noyau
tronquée : plus de points dans le passé que dans le futur sont utilisés
pour estimer les derniers coefficients. C'est donc également une source
de révision au fur et à mesure que des nouveaux points seront connus.
Même si des méthodes optimales existent pour minimiser les erreurs
d'estimation des coefficients en temps réel \autocite[voir par
exemple][]{FengSchafer2021}, cela devrait ici avoir peu d'impact car un
modèle très simple est ici utilisé pour estimer les coefficients
(approximation de la fonction \(\alpha\) par une constante).

Un autre inconvénient de ces méthodes est que tous les coefficients
varient dans le temps alors que dans certains cas on peut supposer la
relation constante.

\subsection{Régression avec coefficients stochastiques (modélisation
espace-état)}\label{sec-ssm}

La modélisation espace-état est une méthodologie générale permettant de
traiter un grand nombre de problèmes de séries temporelles. Dans cette
approche, on suppose que tout modèle est déterminé par une série de
vecteurs non observés \(\bf \alpha_1,\dots,\bf\alpha_T\) associés aux
observations \(y_1,\dots,y_T\), la relation entre \(\alpha_t\) et
\(y_t\) étant spécifiée par le modèle espace-état. Ces modèles sont
largement décrits dans la littérature, notamment par
\textcite{durbinkoopman}. Dans cette étude, nous nous placerons dans un
cadre simplifié des modèles linéaires gaussiens appliqués aux
régressions linéaires. Les modèles sont déterminés par un ensemble de
deux équations : \[
\begin{cases}
y_t={\bf X_t}\bf\alpha_t+\varepsilon_t,\quad&\varepsilon_t\sim\mathcal N(0,\sigma^2)\\
\bf\alpha_{t+1}=\bf\alpha_t+\bf\eta_t,\quad&\bf\eta_t\sim\mathcal N(\bf 0,\bf\Sigma)
\end{cases},\text{ avec }\eta_t\text{ et }\varepsilon_t\text{ indépendants.}
\] La première équation est l'équation d'observation (\emph{observation
equation}), la seconde l'équation d'état (\emph{state equation}) et
\(\bf\alpha_t\) le vecteur d'états (\emph{state vector}).

Dans cette étude, la matrice de variance-covariance \(\bf\Sigma\) est
supposée diagonale : la dynamique d'évolution des coefficients d'une
variable est donc indépendante de la dynamique d'évolution des autres
variables. Lorsque des contraintes entre les différents coefficients
existent, des spécifications différentes de la matrice de
variance-covariance \(\bf\Sigma\) peuvent être faites : c'est par
exemple ce qui a été fait par \textcite{abs2006} pour estimer des
coefficients jours ouvrables variant dans le temps. Chaque coefficient
suivant une marche aléatoire, nous appelons cette méthode \emph{modèle
de régression avec coefficients stochastiques}.

On retrouve le cas de la régression linéaire lorsque \(\bf\Sigma=\bf 0\)
puisque dans ce cas tous les \(\alpha_t\) sont égaux.

Ces modèles sont implémentés dans la fonction \texttt{tvCoef::ssm\_lm()}
qui prend en entrée un modèle de régression linéaire. Elle s'appuie sur
le package \texttt{rjd3sts} \autocite{rjd3sts} qui permet d'implémenter
très facilement les modèles espace-état sans devoir écrire explicitement
le modèle. Par défaut les variances du vecteur d'états (\(\bf \Sigma\))
ne sont pas estimées et sont fixées à 0 : on retrouve donc les
coefficients estimés par régression linéaire.

\begin{Shaded}
\begin{Highlighting}[]
\NormalTok{ssm }\OtherTok{\textless{}{-}} \FunctionTok{ssm\_lm}\NormalTok{(reg\_lin)}
\FunctionTok{summary}\NormalTok{(ssm)}
\end{Highlighting}
\end{Shaded}

\begin{verbatim}
Summary of time-varying estimated coefficients (smoothing): 
        (Intercept) bc_fr_m1 diff_bc_fr_m1      noise
Min.         0.4472  0.02047       0.04423 -1.301e+00
1st Qu.      0.4472  0.02047       0.04423 -2.388e-01
Median       0.4472  0.02047       0.04423  2.808e-02
Mean         0.4472  0.02047       0.04423  2.933e-16
3rd Qu.      0.4472  0.02047       0.04423  2.449e-01
Max.         0.4472  0.02047       0.04423  9.429e-01
\end{verbatim}

L'estimation des hyperparamètres (variances des bruits blancs \(\eta_t\)
et \(\varepsilon_t\)) est faite par maximum de vraisemblance, et
différentes méthodes existent pour initialiser les modèles (calculer
\(\bf \alpha_1\)). Pour plus de détails voir par exemple
\textcite{durbinkoopman}. Le filtre de Kalman permet ensuite de calculer
tous les coefficients. Parmi les paramètres calculés, les deux
principaux sont :

\begin{enumerate}
\def\labelenumi{\arabic{enumi}.}
\tightlist
\item
  Les états lissés (\emph{smoothed states})
  \(\E{\alpha_t|y_1,\dots,y_n}\) : il s'agit de l'estimation des états
  (\(\bf\alpha_t\)) en utilisant toute l'information disponible. Dans le
  cadre de la régression linéaire, les états lissés sont donc constants
  sur toutes les dates et correspondent aux coefficients estimés en
  utilisant l'ensemble des données disponibles :
\end{enumerate}

\begin{Shaded}
\begin{Highlighting}[]
\FunctionTok{window}\NormalTok{(ssm}\SpecialCharTok{$}\NormalTok{smoothed\_states, }\AttributeTok{start =} \DecValTok{2019}\NormalTok{)}
\end{Highlighting}
\end{Shaded}

\begin{verbatim}
        (Intercept)   bc_fr_m1 diff_bc_fr_m1       noise
2019 Q1   0.4472074 0.02047286    0.04422754  0.24369089
2019 Q2   0.4472074 0.02047286    0.04422754 -0.04985071
2019 Q3   0.4472074 0.02047286    0.04422754 -0.42305103
2019 Q4   0.4472074 0.02047286    0.04422754 -1.00671428
\end{verbatim}

\begin{enumerate}
\def\labelenumi{\arabic{enumi}.}
\setcounter{enumi}{1}
\tightlist
\item
  Les états filtrés (\emph{filtered states})
  \(\E{\alpha_t|y_1,\dots,y_{t-1}}\) : il s'agit de l'estimation des
  états (\(\bf\alpha_t\)) en utilisant l'information disponible jusqu'à
  la date précédente. Dans le cadre de la régression linéaire, cela
  correspond aux coefficients estimés hors échantillon : la valeur des
  états filtrés en 2010T2 correspond aux coefficients estimés en
  utilisant les données jusqu'au 2010T1. Ils permettent donc d'avoir une
  estimation des prévisions hors échantillon du modèle.
\end{enumerate}

\begin{Shaded}
\begin{Highlighting}[]
\FunctionTok{round}\NormalTok{(}\FunctionTok{window}\NormalTok{(ssm}\SpecialCharTok{$}\NormalTok{filtering\_states, }\AttributeTok{start =} \FunctionTok{c}\NormalTok{(}\DecValTok{2010}\NormalTok{, }\DecValTok{2}\NormalTok{), }\AttributeTok{end =} \FunctionTok{c}\NormalTok{(}\DecValTok{2010}\NormalTok{, }\DecValTok{2}\NormalTok{)), }\DecValTok{6}\NormalTok{)}
\end{Highlighting}
\end{Shaded}

\begin{verbatim}
        (Intercept) bc_fr_m1 diff_bc_fr_m1 noise
2010 Q2    0.493822 0.021962      0.049506     0
\end{verbatim}

\begin{Shaded}
\begin{Highlighting}[]
\FunctionTok{round}\NormalTok{(}\FunctionTok{coef}\NormalTok{(}\FunctionTok{dynlm}\NormalTok{(}
  \AttributeTok{formula =}\NormalTok{ growth\_gdp }\SpecialCharTok{\textasciitilde{}}\NormalTok{ bc\_fr\_m1 }\SpecialCharTok{+}\NormalTok{ diff\_bc\_fr\_m1,}
  \AttributeTok{data =} \FunctionTok{window}\NormalTok{(data\_gdp, }\AttributeTok{start =} \DecValTok{1980}\NormalTok{, }\AttributeTok{end =} \FunctionTok{c}\NormalTok{(}\DecValTok{2010}\NormalTok{,}\DecValTok{1}\NormalTok{))}
\NormalTok{)), }\DecValTok{6}\NormalTok{)}
\end{Highlighting}
\end{Shaded}

\begin{verbatim}
  (Intercept)      bc_fr_m1 diff_bc_fr_m1 
     0.493822      0.021962      0.049506 
\end{verbatim}

Lorsque les variances sont estimées, les états filtrés ne correspondent
pas exactement à des estimations hors échantillon car les
hyperparamètres restent fixés (variances \(\bf\Sigma\) et
initialisation). Les estimations hors échantillon peuvent être calculées
en utilisant la fonction \texttt{tvCoef::ssm\_lm\_oos()}.

Pour faciliter l'estimation des variances \(\bf\Sigma,\) le modèle est
souvent reparamétré : \[
\begin{cases}
y_t={\bf X_t}\bf\alpha_t+\varepsilon_t,\quad&\varepsilon_t\sim\mathcal N(0,\sigma^2)\\
\bf\alpha_{t+1}=\bf\alpha_t+\bf\eta_t,\quad&\bf\eta_t\sim\mathcal N(\bf 0,\sigma^2\bf Q)
\end{cases},\text{ avec }\eta_t\text{ et }\varepsilon_t\text{ indépendants.}
\] Les variances sont donc définies à un facteur multiplicatif près et
une estimation en deux étapes est faite : la vraisemblance est dite
\emph{concentrée}. C'est ce qui est utilisé par défaut dans
\texttt{tvCoef::ssm\_lm()}. Dans notre exemple, l'erreur standard de la
régression (\emph{residual standar error}) peut se calculer de la façon
suivante :

\begin{Shaded}
\begin{Highlighting}[]
\FunctionTok{sqrt}\NormalTok{(ssm}\SpecialCharTok{$}\NormalTok{parameters}\SpecialCharTok{$}\NormalTok{parameters }\SpecialCharTok{*}\NormalTok{ ssm}\SpecialCharTok{$}\NormalTok{parameters}\SpecialCharTok{$}\NormalTok{scaling)}
\end{Highlighting}
\end{Shaded}

\begin{verbatim}
(Intercept).var       noise.var 
      0.0000000       0.3888042 
\end{verbatim}

\begin{Shaded}
\begin{Highlighting}[]
\FunctionTok{summary}\NormalTok{(reg\_lin)}\SpecialCharTok{$}\NormalTok{sigma}
\end{Highlighting}
\end{Shaded}

\begin{verbatim}
[1] 0.3888042
\end{verbatim}

Afin d'estimer les variances associées à l'équation d'état, il faut
utiliser les paramètres \texttt{fixed\_var\_intercept\ =\ FALSE} et
\texttt{fixed\_var\_trend\ =\ FALSE}. Même si la valeur des variances
(\texttt{var\_intercept} et \texttt{var\_variables} qui valent 0 par
défaut) n'aura aucun effet sur les variances finales estimées, il est
parfois nécessaire de modifier ces valeurs afin d'éviter une erreur dans
l'optimisation. Sur notre modèle, l'optimisation conduit à garder fixe
le coefficient associé au climat des affaires en niveau (variance nulle)
mais considère que les autres variables varient dans le temps~:

\begin{Shaded}
\begin{Highlighting}[]
\NormalTok{ssm }\OtherTok{\textless{}{-}} \FunctionTok{ssm\_lm}\NormalTok{(reg\_lin, }
              \AttributeTok{fixed\_var\_intercept =} \ConstantTok{FALSE}\NormalTok{, }
              \AttributeTok{fixed\_var\_variables =} \ConstantTok{FALSE}\NormalTok{,}
              \AttributeTok{var\_intercept =} \FloatTok{0.01}\NormalTok{,}
              \AttributeTok{var\_variables =} \FloatTok{0.01}\NormalTok{)}
\FunctionTok{sqrt}\NormalTok{(ssm}\SpecialCharTok{$}\NormalTok{parameters}\SpecialCharTok{$}\NormalTok{parameters }\SpecialCharTok{*}\NormalTok{ ssm}\SpecialCharTok{$}\NormalTok{parameters}\SpecialCharTok{$}\NormalTok{scaling)}
\end{Highlighting}
\end{Shaded}

\begin{verbatim}
  (Intercept).var      bc_fr_m1.var diff_bc_fr_m1.var         noise.var 
      0.019311187       0.000000000       0.008051459       0.352041285 
\end{verbatim}

\begin{remark}
Dans la version actuelle de \texttt{tvCoef}, les retards de la variable
endogène (à prévoir) ne sont pas modélisés correctement. En effet, dans
ce cas il faudrait utiliser une modélisation différente afin de prendre
en compte la relation entre la variable endogène et les retards.
\end{remark}

La figure~\ref{fig-coef-ssm} montre les coefficients estimés avec toutes
les méthodes présentées dans ce papier. Pour toutes les méthodes, le
coefficient du climat des affaires en niveau est stable dans le temps
(coefficients estimés compris entre 0,020 et 0,023, même pour la
régression locale les différences ne sont pas significatives). En
revanche, les résultats de la régression avec coefficients stochastiques
sont sensiblement différents pour le coefficient associé à la variation
du climat des affaires, avec des périodes où le coefficient est plus
faible que pour les autres méthodes (avant 1983, entre 1995 et 1999 et
après 2011) et d'autres où il est plus élevé (notamment pendant la crise
financière).

\begin{figure}

\caption{\label{fig-coef-ssm}Coefficients estimés par régression
linéaire, régression par morceaux, régression locale (avec \(b=0,75\))
et régression avec coefficients stochastiques (modélisation
espace-état).}

\centering{

\includegraphics{DT-tvcoef_files/figure-pdf/ffig-coef-ssm-1.pdf}

\footnotesize\raggedright

\textbf{Note} : les échelles sont différentes entre les différents
graphiques.\\
\textbf{Source} : Insee (PIB et climat des affaires France entre 1980 et
2019 téléchargés le 15 mars 2024), calculs de l'auteur.

}

\end{figure}%

La table~\ref{tbl-res-model-pib} compare la qualité prédictive des
différents modèles dans l'échantillon (en utilisant toutes les données
pour estimer les paramètres des modèles) et hors échantillon
(reproduction du processus de prévision en estimant de manière récursive
les modèles jusqu'à la date \(t\) pour calculer les prévisions à la date
\(t+1\)). C'est la régression avec coefficients stochastiques
(modélisation espace-état) qui minimise les erreurs de prévision (dans
et hors échantillon), suivie de la régression par morceaux. La
régression locale a une erreur hors échantillon plus élevée notamment du
fait des instabilités sur l'estimation de la fenêtre. Le test de
Diebold-Mariano (voir notamment \textcite{DMtest}), implémenté dans la
fonction \texttt{forecast::dm.test()} \autocite{forecastR}, permet de
tester si cette différence est significative. Dans et hors échantillon,
la régression avec coefficients stochastiques a des erreurs de prévision
significativement plus petites que la régression linéaire (p-valeurs de
0,00 et 0,05). Dans l'échantillon elles sont également significativement
plus petites que celles de la régression par morceaux (p-valeur de 0,00)
mais la différence n'est pas significative hors échantillon (p-valeur de
0,09). Sur les périodes récentes, du fait du coefficient sur le climat
des affaires en différence, l'interprétation économique et les
prévisions sont différentes.

\begin{Shaded}
\begin{Highlighting}[]
\NormalTok{oos\_ssm }\OtherTok{\textless{}{-}} \FunctionTok{ssm\_lm\_oos}\NormalTok{(reg\_lin, }\AttributeTok{fixed\_var\_intercept =} \ConstantTok{FALSE}\NormalTok{, }
                      \AttributeTok{fixed\_var\_variables =} \ConstantTok{FALSE}\NormalTok{,}
                      \AttributeTok{date =} \DecValTok{70}\NormalTok{)}

\NormalTok{res\_is }\OtherTok{\textless{}{-}} \FunctionTok{ts.union}\NormalTok{(}
  \FunctionTok{ts}\NormalTok{(}\FunctionTok{residuals}\NormalTok{(reg\_lin), }\AttributeTok{end =} \FunctionTok{c}\NormalTok{(}\DecValTok{2019}\NormalTok{,}\DecValTok{4}\NormalTok{), }\AttributeTok{frequency =} \DecValTok{4}\NormalTok{),}
  \FunctionTok{residuals}\NormalTok{(reg\_morc),}
  \FunctionTok{ts}\NormalTok{(}\FunctionTok{residuals}\NormalTok{(reg\_loc), }\AttributeTok{end =} \FunctionTok{c}\NormalTok{(}\DecValTok{2019}\NormalTok{,}\DecValTok{4}\NormalTok{), }\AttributeTok{frequency =} \DecValTok{4}\NormalTok{),}
  \FunctionTok{residuals}\NormalTok{(ssm)[, }\StringTok{"smoothed"}\NormalTok{]}
\NormalTok{)}
\NormalTok{res\_oos }\OtherTok{\textless{}{-}} \FunctionTok{ts.union}\NormalTok{(}
\NormalTok{  oos\_lm}\SpecialCharTok{$}\NormalTok{residuals,}
\NormalTok{  oos\_reg\_morc}\SpecialCharTok{$}\NormalTok{residuals,}
  \FunctionTok{ts}\NormalTok{(oos\_reg\_loc}\SpecialCharTok{$}\NormalTok{residuals, }\AttributeTok{end =} \FunctionTok{c}\NormalTok{(}\DecValTok{2019}\NormalTok{,}\DecValTok{4}\NormalTok{), }\AttributeTok{frequency =} \DecValTok{4}\NormalTok{),}
\NormalTok{  oos\_ssm}\SpecialCharTok{$}\NormalTok{oos\_noise}
\NormalTok{)}
\NormalTok{res\_oos }\OtherTok{\textless{}{-}} \FunctionTok{window}\NormalTok{(res\_oos, }\AttributeTok{start =} \DecValTok{2003}\NormalTok{)}
\CommentTok{\# (H0) : Reg. coef stochastique meilleure que modèle linéaire}
\NormalTok{forecast}\SpecialCharTok{::}\FunctionTok{dm.test}\NormalTok{(res\_is[, }\DecValTok{1}\NormalTok{], res\_is[, }\DecValTok{4}\NormalTok{], }\StringTok{"greater"}\NormalTok{)}
\end{Highlighting}
\end{Shaded}

\begin{verbatim}

    Diebold-Mariano Test

data:  res_is[, 1]res_is[, 4]
DM = 3.2675, Forecast horizon = 1, Loss function power = 2, p-value =
0.0006646
alternative hypothesis: greater
\end{verbatim}

\begin{Shaded}
\begin{Highlighting}[]
\CommentTok{\# (H0) : Reg. coef stochastique meilleure que régression par morceaux}
\NormalTok{forecast}\SpecialCharTok{::}\FunctionTok{dm.test}\NormalTok{(res\_is[, }\DecValTok{2}\NormalTok{], res\_is[, }\DecValTok{4}\NormalTok{], }\StringTok{"greater"}\NormalTok{)}
\end{Highlighting}
\end{Shaded}

\begin{verbatim}

    Diebold-Mariano Test

data:  res_is[, 2]res_is[, 4]
DM = 3.0553, Forecast horizon = 1, Loss function power = 2, p-value =
0.001319
alternative hypothesis: greater
\end{verbatim}

\begin{Shaded}
\begin{Highlighting}[]
\CommentTok{\# (H0) : Reg. coef stochastique meilleure que modèle linéaire}
\NormalTok{forecast}\SpecialCharTok{::}\FunctionTok{dm.test}\NormalTok{(res\_oos[, }\DecValTok{1}\NormalTok{], res\_oos[, }\DecValTok{4}\NormalTok{], }\StringTok{"greater"}\NormalTok{)}
\end{Highlighting}
\end{Shaded}

\begin{verbatim}

    Diebold-Mariano Test

data:  res_oos[, 1]res_oos[, 4]
DM = 1.6199, Forecast horizon = 1, Loss function power = 2, p-value =
0.05498
alternative hypothesis: greater
\end{verbatim}

\begin{Shaded}
\begin{Highlighting}[]
\CommentTok{\# (H0) : Reg. coef stochastique meilleure que régression par morceaux}
\NormalTok{forecast}\SpecialCharTok{::}\FunctionTok{dm.test}\NormalTok{(res\_oos[, }\DecValTok{2}\NormalTok{], res\_oos[, }\DecValTok{4}\NormalTok{], }\StringTok{"greater"}\NormalTok{)}
\end{Highlighting}
\end{Shaded}

\begin{verbatim}

    Diebold-Mariano Test

data:  res_oos[, 2]res_oos[, 4]
DM = 1.3245, Forecast horizon = 1, Loss function power = 2, p-value =
0.09492
alternative hypothesis: greater
\end{verbatim}

\begin{longtable}[]{@{}
  >{\raggedright\arraybackslash}p{(\columnwidth - 4\tabcolsep) * \real{0.6000}}
  >{\centering\arraybackslash}p{(\columnwidth - 4\tabcolsep) * \real{0.2105}}
  >{\centering\arraybackslash}p{(\columnwidth - 4\tabcolsep) * \real{0.1895}}@{}}

\caption{\label{tbl-res-model-pib}Erreurs quadratiques moyennes des
erreurs de prévision entre les différentes méthodes.}

\tabularnewline

\toprule\noalign{}
\begin{minipage}[b]{\linewidth}\raggedright
\end{minipage} & \begin{minipage}[b]{\linewidth}\centering
Dans l'échantillon
\end{minipage} & \begin{minipage}[b]{\linewidth}\centering
Hors échantillon
\end{minipage} \\
\midrule\noalign{}
\endhead
\bottomrule\noalign{}
\endlastfoot
Régression linéaire & 0,39 & 0,43 \\
Régression par morceaux & 0,36 & 0,40 \\
Régression locale & 0,36 & 0,42 \\
Régression avec coefficients stochastiques (espace-état) & 0,34 &
0,39 \\

\end{longtable}

{\footnotesize\raggedright

\textbf{Note} : les prévisions dans l'échantillon sont calculées en
estimant les modèles à partir des données disponibles entre 1980T1 et
2019T4. Les prévisions hors échantillon sont calculées à partir de
2003T1 : la première prévision (2003T1) correspond à celle que l'on
aurait eu en estimant les modèles à partir des données disponibles
jusqu'en 2002T4 (trimestre précédente).\\
\textbf{Source} : Insee (PIB et climat des affaires France entre 1980 et
2019 téléchargés le 15 mars 2024), calculs de l'auteur.

}

\subsection{Prise en compte de la période du COVID-19 et
prévision}\label{prise-en-compte-de-la-puxe9riode-du-covid-19-et-pruxe9vision}

Dans les sections précédentes, les modèles ont été estimés jusqu'en
2019T4 dans le but de simplifier la présentation des modèles. Toutefois,
si l'on veut effectuer de la prévision sur les périodes récentes, il est
indispensable de prendre en compte la période du COVID-19. Cela se fait
généralement en ajoutant, dans le modèle de prévision, des variables
explicatives modélisant les chocs de cette période. La méthode la plus
simple consiste à ajouter des indicatrices sur les trimestres concernés
(ici l'année 2020 et le trimestre 2021T3)\footnote{ D'autres
  spécifications pourraient être utilisées, comme par exemple l'ajout
  d'indicatrices sur l'ensemble des années 2020 et 2021 ou uniquement
  sur l'année 2020. Ajouter ou retirer d'autres indicatrices peut
  sensiblement changer les résultats des coefficients estimés sur la fin
  de la période, notamment du fait du faible recul temporel que l'on a
  après le COVID-19 : le choix dépend aussi des hypothèses économiques
  que l'on fait sur le modèle utilisé (quels sont les trimestres qui
  sont atypiques et ceux dont les évolutions relèvent de la conjoncture
  ?). Ici nous avons choisi de n'utiliser des indicatrices que pour
  l'année 2020 et le trimestre 2021T3 car les indicatrices des autres
  trimestres de 2021 ne sont pas significatives, que l'année 2020 est
  fortement heurtée par la crise du COVID-19 et que la forte croissance
  de 2021T3 peut s'expliquer par un contre-coup des mesures de
  confinement de 2021T2.}~:

\begin{Shaded}
\begin{Highlighting}[]
\NormalTok{ind }\OtherTok{\textless{}{-}} \FunctionTok{cbind}\NormalTok{(}
  \FunctionTok{time}\NormalTok{(gdp) }\SpecialCharTok{==} \DecValTok{2020}\NormalTok{, }\FunctionTok{time}\NormalTok{(gdp) }\SpecialCharTok{==} \FloatTok{2020.25}\NormalTok{, }
  \FunctionTok{time}\NormalTok{(gdp) }\SpecialCharTok{==} \FloatTok{2020.5}\NormalTok{, }\FunctionTok{time}\NormalTok{(gdp) }\SpecialCharTok{==} \FloatTok{2020.75}\NormalTok{,}
  \FunctionTok{time}\NormalTok{(gdp) }\SpecialCharTok{==} \FloatTok{2021.5}
\NormalTok{)}
\NormalTok{ind }\OtherTok{\textless{}{-}} \FunctionTok{ts}\NormalTok{(}\FunctionTok{apply}\NormalTok{(ind,}\DecValTok{2}\NormalTok{, as.numeric), }\AttributeTok{start =} \FunctionTok{start}\NormalTok{(gdp), }\AttributeTok{frequency =} \DecValTok{4}\NormalTok{)}
\FunctionTok{colnames}\NormalTok{(ind) }\OtherTok{\textless{}{-}} \FunctionTok{c}\NormalTok{(}\FunctionTok{sprintf}\NormalTok{(}\StringTok{"ind2020T\%i"}\NormalTok{, }\DecValTok{1}\SpecialCharTok{:}\DecValTok{4}\NormalTok{), }\StringTok{"ind2021T3"}\NormalTok{)}
\NormalTok{data\_covid }\OtherTok{\textless{}{-}} \FunctionTok{ts.union}\NormalTok{(gdp, ind)}
\FunctionTok{colnames}\NormalTok{(data\_covid) }\OtherTok{\textless{}{-}} \FunctionTok{c}\NormalTok{(}\FunctionTok{colnames}\NormalTok{(gdp), }\FunctionTok{colnames}\NormalTok{(ind))}
\CommentTok{\# Renormalisation à 0 du climat des affaires}
\NormalTok{bc\_variables }\OtherTok{\textless{}{-}} \FunctionTok{c}\NormalTok{(}\StringTok{"bc\_fr\_m1"}\NormalTok{, }\StringTok{"bc\_fr\_m2"}\NormalTok{, }\StringTok{"bc\_fr\_m3"}\NormalTok{)}
\NormalTok{data\_covid[, bc\_variables] }\OtherTok{\textless{}{-}}\NormalTok{ data\_covid[, bc\_variables] }\SpecialCharTok{{-}} \DecValTok{100}

\NormalTok{reg\_lin\_covid }\OtherTok{\textless{}{-}} \FunctionTok{dynlm}\NormalTok{(}
  \AttributeTok{formula =}\NormalTok{ growth\_gdp }\SpecialCharTok{\textasciitilde{}}\NormalTok{ bc\_fr\_m1 }\SpecialCharTok{+}\NormalTok{ diff\_bc\_fr\_m1 }\SpecialCharTok{+}
\NormalTok{    ind2020T1 }\SpecialCharTok{+}\NormalTok{ ind2020T2 }\SpecialCharTok{+}\NormalTok{ ind2020T3 }\SpecialCharTok{+}\NormalTok{ ind2020T4 }\SpecialCharTok{+} 
\NormalTok{    ind2021T3,}
  \AttributeTok{data =} \FunctionTok{window}\NormalTok{(data\_covid, }\AttributeTok{start =} \DecValTok{1980}\NormalTok{)}
\NormalTok{)}
\FunctionTok{summary}\NormalTok{(reg\_lin\_covid)}
\end{Highlighting}
\end{Shaded}

\begin{verbatim}

Time series regression with "ts" data:
Start = 1980(1), End = 2023(4)

Call:
dynlm(formula = growth_gdp ~ bc_fr_m1 + diff_bc_fr_m1 + ind2020T1 + 
    ind2020T2 + ind2020T3 + ind2020T4 + ind2021T3, data = window(data_covid, 
    start = 1980))

Residuals:
     Min       1Q   Median       3Q      Max 
-1.28405 -0.24352  0.02404  0.24979  0.95442 

Coefficients:
               Estimate Std. Error t value Pr(>|t|)    
(Intercept)    0.434848   0.029527  14.727  < 2e-16 ***
bc_fr_m1       0.019944   0.003101   6.431 1.26e-09 ***
diff_bc_fr_m1  0.045651   0.007251   6.296 2.57e-09 ***
ind2020T1     -5.860936   0.387628 -15.120  < 2e-16 ***
ind2020T2     -9.849472   0.575971 -17.101  < 2e-16 ***
ind2020T3     15.297386   0.503486  30.383  < 2e-16 ***
ind2020T4     -0.839406   0.388384  -2.161 0.032090 *  
ind2021T3      1.525800   0.405567   3.762 0.000232 ***
---
Signif. codes:  0 '***' 0.001 '**' 0.01 '*' 0.05 '.' 0.1 ' ' 1

Residual standard error: 0.386 on 168 degrees of freedom
  (0 observation effacée parce que manquante)
Multiple R-squared:  0.9551,    Adjusted R-squared:  0.9532 
F-statistic: 510.1 on 7 and 168 DF,  p-value: < 2.2e-16
\end{verbatim}

Pour la construction des autres modèles, nous gardons certains
paramètres estimés en utilisant les données avant la période du
COVID-19, cette dernière pouvant biaiser les résultats. Ainsi~:

\begin{itemize}
\tightlist
\item
  Pour la régression par morceaux, la date de rupture retenue est
  toujours 2000T3 (contre 2017T2 en utilisant les données après 2020).
\end{itemize}

\begin{Shaded}
\begin{Highlighting}[]
\NormalTok{bp\_covid }\OtherTok{\textless{}{-}} \FunctionTok{breakpoints}\NormalTok{(reg\_lin\_covid)}
\FunctionTok{c}\NormalTok{(}\FunctionTok{breakdates}\NormalTok{(bp), }\FunctionTok{breakdates}\NormalTok{(bp\_covid))}
\end{Highlighting}
\end{Shaded}

\begin{verbatim}
[1] 2000.50 2017.25
\end{verbatim}

\begin{itemize}
\tightlist
\item
  Pour la régression locale, la fenêtre utilisée est 0,75 (proche de la
  fenêtre de 0,71 en utilisant les données après 2020).
\end{itemize}

\begin{Shaded}
\begin{Highlighting}[]
\NormalTok{bw\_covid }\OtherTok{\textless{}{-}} \FunctionTok{bw}\NormalTok{(}\FunctionTok{window}\NormalTok{(}
\NormalTok{  data\_covid[, }\FunctionTok{c}\NormalTok{(}\StringTok{"bc\_fr\_m1"}\NormalTok{, }\StringTok{"diff\_bc\_fr\_m1"}\NormalTok{,}
                 \StringTok{"ind2020T1"}\NormalTok{, }\StringTok{"ind2020T2"}\NormalTok{, }
                 \StringTok{"ind2020T3"}\NormalTok{, }\StringTok{"ind2020T4"}\NormalTok{, }
                 \StringTok{"ind2021T3"}\NormalTok{)], }\AttributeTok{start =} \DecValTok{1980}\NormalTok{, }\AttributeTok{end =} \FunctionTok{c}\NormalTok{(}\DecValTok{2023}\NormalTok{, }\DecValTok{4}\NormalTok{)), }
  \FunctionTok{window}\NormalTok{(data\_covid[, }\StringTok{"growth\_gdp"}\NormalTok{], }\AttributeTok{start =} \DecValTok{1980}\NormalTok{, }\AttributeTok{end =} \FunctionTok{c}\NormalTok{(}\DecValTok{2023}\NormalTok{, }\DecValTok{4}\NormalTok{))}
\NormalTok{)}
\FunctionTok{c}\NormalTok{(reg\_loc}\SpecialCharTok{$}\NormalTok{bw, bw\_covid)}
\end{Highlighting}
\end{Shaded}

\begin{verbatim}
[1] 0.7481989 0.7068480
\end{verbatim}

\begin{itemize}
\tightlist
\item
  Pour la régression avec coefficients stochastiques (modélisation
  espace-état), les coefficients associés aux indicatrices sont fixés
  (variance nulle, sinon ils sont considérés comme évoluant dans le
  temps) et le coefficient du climat des affaires en niveau est toujours
  considéré comme fixe. Toutefois les variances des autres coefficients
  sont de nouveau estimées.
\end{itemize}

\begin{Shaded}
\begin{Highlighting}[]
\NormalTok{ssm\_covid }\OtherTok{\textless{}{-}} \FunctionTok{ssm\_lm}\NormalTok{(}
\NormalTok{  reg\_lin\_covid, }
  \AttributeTok{fixed\_var\_intercept =} \ConstantTok{FALSE}\NormalTok{,}
  \AttributeTok{fixed\_var\_variables =} \ConstantTok{FALSE}\NormalTok{,}
  \AttributeTok{var\_intercept =} \FloatTok{0.01}\NormalTok{,}
  \AttributeTok{var\_variables =} \FloatTok{0.01}
\NormalTok{)}
\FunctionTok{sqrt}\NormalTok{(ssm\_covid}\SpecialCharTok{$}\NormalTok{parameters}\SpecialCharTok{$}\NormalTok{parameters }\SpecialCharTok{*}\NormalTok{ ssm\_covid}\SpecialCharTok{$}\NormalTok{parameters}\SpecialCharTok{$}\NormalTok{scaling)}
\end{Highlighting}
\end{Shaded}

\begin{verbatim}
  (Intercept).var      bc_fr_m1.var diff_bc_fr_m1.var     ind2020T1.var 
     1.860970e-02      0.000000e+00      6.833696e-03      2.615608e+06 
    ind2020T2.var     ind2020T3.var     ind2020T4.var     ind2021T3.var 
     8.891591e-01      0.000000e+00      2.092569e-01      5.071114e-01 
        noise.var 
     3.509538e-01 
\end{verbatim}

Les modèles sont donc estimés avec le code suivant :

\begin{Shaded}
\begin{Highlighting}[]
\NormalTok{reg\_morc\_covid }\OtherTok{\textless{}{-}} \FunctionTok{piece\_reg}\NormalTok{(}
\NormalTok{  reg\_lin\_covid, }\AttributeTok{break\_dates =} \FloatTok{2000.5}\NormalTok{,}
  \CommentTok{\# Les indicatrices ne sont pas découpées}
  \AttributeTok{fixed\_var =} \DecValTok{4}\SpecialCharTok{:}\DecValTok{8}\NormalTok{)}
\NormalTok{reg\_loc\_covid }\OtherTok{\textless{}{-}}\NormalTok{ tvReg}\SpecialCharTok{::}\FunctionTok{tvLM}\NormalTok{(}
  \AttributeTok{formula =}\NormalTok{ growth\_gdp }\SpecialCharTok{\textasciitilde{}}\NormalTok{ bc\_fr\_m1 }\SpecialCharTok{+}\NormalTok{ diff\_bc\_fr\_m1 }\SpecialCharTok{+}
\NormalTok{    ind2020T1 }\SpecialCharTok{+}\NormalTok{ ind2020T2 }\SpecialCharTok{+}\NormalTok{ ind2020T3 }\SpecialCharTok{+}\NormalTok{ ind2020T4 }\SpecialCharTok{+}\NormalTok{ ind2021T3,}
  \AttributeTok{data =} \FunctionTok{window}\NormalTok{(data\_covid, }\AttributeTok{start =} \DecValTok{1980}\NormalTok{),}
  \CommentTok{\# On reprend l\textquotesingle{}ancienne fenêtre}
  \AttributeTok{bw =}\NormalTok{ reg\_loc}\SpecialCharTok{$}\NormalTok{bw }
\NormalTok{)}
\NormalTok{ssm\_covid }\OtherTok{\textless{}{-}} \FunctionTok{ssm\_lm}\NormalTok{(}
\NormalTok{  reg\_lin\_covid, }
  \AttributeTok{fixed\_var\_intercept =} \ConstantTok{FALSE}\NormalTok{,}
  \CommentTok{\# On fixe les coefficients des indicatrices}
  \CommentTok{\# et le coefficient du climat des affaires en niveau }
  \CommentTok{\# (sinon il varie dans le temps)}
  \AttributeTok{fixed\_var\_variables =} \FunctionTok{c}\NormalTok{(}\FunctionTok{c}\NormalTok{(}\ConstantTok{TRUE}\NormalTok{, }\ConstantTok{FALSE}\NormalTok{), }\FunctionTok{rep}\NormalTok{(}\ConstantTok{TRUE}\NormalTok{, }\DecValTok{5}\NormalTok{)),}
  \AttributeTok{var\_intercept =} \FloatTok{0.01}\NormalTok{,}
  \AttributeTok{var\_variables =} \FunctionTok{c}\NormalTok{(}\DecValTok{0}\NormalTok{, }\FloatTok{0.01}\NormalTok{, }\FunctionTok{rep}\NormalTok{(}\DecValTok{0}\NormalTok{, }\DecValTok{5}\NormalTok{))}
\NormalTok{)}

\NormalTok{coef\_morc\_covid }\OtherTok{\textless{}{-}} \FunctionTok{coef}\NormalTok{(reg\_morc\_covid)}
\NormalTok{coef\_lin\_covid }\OtherTok{\textless{}{-}} \FunctionTok{ts}\NormalTok{(}\FunctionTok{matrix}\NormalTok{(}\FunctionTok{coef}\NormalTok{(reg\_lin\_covid), }\AttributeTok{nrow =} \DecValTok{1}\NormalTok{), }
               \AttributeTok{start =} \FunctionTok{start}\NormalTok{(coef\_morc\_covid),}
               \AttributeTok{end =} \FunctionTok{end}\NormalTok{(coef\_morc\_covid),}
               \AttributeTok{frequency =} \FunctionTok{frequency}\NormalTok{(coef\_morc\_covid))}
\FunctionTok{colnames}\NormalTok{(coef\_lin\_covid) }\OtherTok{\textless{}{-}} \FunctionTok{names}\NormalTok{(}\FunctionTok{coef}\NormalTok{(reg\_lin\_covid))}
\NormalTok{coef\_reg\_loc\_covid }\OtherTok{\textless{}{-}} \FunctionTok{ts}\NormalTok{(}\FunctionTok{coef}\NormalTok{(reg\_loc\_covid), }\AttributeTok{start =} \DecValTok{1980}\NormalTok{, }\AttributeTok{frequency =} \DecValTok{4}\NormalTok{)}
\NormalTok{coef\_ssm\_covid }\OtherTok{\textless{}{-}} \FunctionTok{coef}\NormalTok{(ssm\_covid)}
\end{Highlighting}
\end{Shaded}

La figure~\ref{fig-coef-covid} montre les coefficients estimés en
prenant en compte les données jusqu'au 2023T4. L'analyse est similaire à
celle de la figure~\ref{fig-coef-ssm} sauf pour la régression avec
coefficients stochastiques sur les dernières années : les variations du
climat des affaires ayant un impact sur la prévision du PIB plus marqué
après 2020, le coefficient associé est plus élevé après cette date mais
aussi lors des années précédentes afin de lisser le passage à ce nouvel
état de l'économie.

\begin{figure}

\caption{\label{fig-coef-covid}Coefficients estimés (hors indicatrices)
par régression linéaire, régression par morceaux, régression locale
(avec \(b=0,75\)) et régression avec coefficients stochastiques
(modélisation espace-état) en utilisant les données jusqu'en 2023T4.}

\centering{

\includegraphics{DT-tvcoef_files/figure-pdf/ffig-coef-covid-1.pdf}

\footnotesize\raggedright

\textbf{Note} : les échelles sont différentes entre les différents
graphiques.\\
\textbf{Source} : Insee (PIB et climat des affaires France entre 1980 et
2023 téléchargés le 15 mars 2024), calculs de l'auteur.

}

\end{figure}%

Dans les données ici utilisées, le taux de croissance trimestriel du PIB
est connu jusqu'au 2023T4 alors que les variables explicatives sont
connues jusqu'au 2024T1 :

\begin{Shaded}
\begin{Highlighting}[]
\FunctionTok{tail}\NormalTok{(data\_covid[, }\FunctionTok{c}\NormalTok{(}\StringTok{"growth\_gdp"}\NormalTok{, }\StringTok{"bc\_fr\_m1"}\NormalTok{, }\StringTok{"diff\_bc\_fr\_m1"}\NormalTok{)], }\DecValTok{2}\NormalTok{)}
\end{Highlighting}
\end{Shaded}

\begin{verbatim}
        growth_gdp bc_fr_m1 diff_bc_fr_m1
2023 Q4 0.05240066     -1.8          -1.9
2024 Q1         NA     -1.4           0.4
\end{verbatim}

Il est donc possible d'effectuer une prévision sur le dernier trimestre
et nous allons maintenant montrer comment procéder. Le plus simple est
d'effectuer une somme pondérée des variables explicatives avec les
coefficients estimés. Dans cet exemple, les variables explicatives sont
directement calculées dans la base de données en entrée et sont donc
faciles à extraire. Lorsque ce n'est pas le cas (par exemple lorsque des
variables retardées ou en différence sont directement calculées dans la
formule de la fonction \texttt{dynlm()}\footnote{ Ce qui aurait pu être
  le cas si le modèle avait été estimé en utilisant le paramètre
  \texttt{formula\ =\ growth\_gdp\ \textasciitilde{}\ bc\_fr\_m1\ +\ diff(bc\_fr\_m1,\ 1)}
  dans la fonction \texttt{dynlm()}.}) il faut alors recalculer toutes
les variables explicatives. La fonction
\texttt{tvCoef::full\_exogeneous\_matrix()} peut aider à effectuer cette
tâche et ajoute également le régresseur associé à la constante (égal à
1) :

\begin{Shaded}
\begin{Highlighting}[]
\NormalTok{data\_prev }\OtherTok{\textless{}{-}} \FunctionTok{full\_exogeneous\_matrix}\NormalTok{(reg\_lin\_covid)}
\CommentTok{\# On extrait le dernier trimestre :}
\NormalTok{der\_period }\OtherTok{\textless{}{-}} \FunctionTok{tail}\NormalTok{(data\_prev, }\DecValTok{1}\NormalTok{)}
\NormalTok{der\_period}
\end{Highlighting}
\end{Shaded}

\begin{verbatim}
        (Intercept) bc_fr_m1 diff_bc_fr_m1 ind2020T1 ind2020T2 ind2020T3
2024 Q1           1     -1.4           0.4         0         0         0
        ind2020T4 ind2021T3
2024 Q1         0         0
\end{verbatim}

\begin{Shaded}
\begin{Highlighting}[]
\CommentTok{\# Transformation en numeric pour éviter des erreurs dues au format ts()}
\NormalTok{der\_period }\OtherTok{\textless{}{-}} \FunctionTok{as.numeric}\NormalTok{(der\_period)}
\NormalTok{prev\_reg\_lin }\OtherTok{\textless{}{-}} \FunctionTok{sum}\NormalTok{(}\FunctionTok{coef}\NormalTok{(reg\_lin\_covid) }\SpecialCharTok{*}\NormalTok{ der\_period)}
\CommentTok{\# Pour les autres méthodes on prend les derniers coefficients estimés}
\NormalTok{prev\_reg\_morc }\OtherTok{\textless{}{-}} \FunctionTok{sum}\NormalTok{(}\FunctionTok{tail}\NormalTok{(}\FunctionTok{coef}\NormalTok{(reg\_morc\_covid), }\DecValTok{1}\NormalTok{) }\SpecialCharTok{*}\NormalTok{ der\_period)}
\NormalTok{prev\_reg\_loc }\OtherTok{\textless{}{-}} \FunctionTok{sum}\NormalTok{(}\FunctionTok{tail}\NormalTok{(}\FunctionTok{coef}\NormalTok{(reg\_loc\_covid), }\DecValTok{1}\NormalTok{) }\SpecialCharTok{*}\NormalTok{ der\_period)}
\NormalTok{prev\_ssm }\OtherTok{\textless{}{-}} \FunctionTok{sum}\NormalTok{(}\FunctionTok{tail}\NormalTok{(}\FunctionTok{coef}\NormalTok{(ssm\_covid), }\DecValTok{1}\NormalTok{) }\SpecialCharTok{*}\NormalTok{ der\_period)}
\CommentTok{\# Ensemble des prévisions pour 2023T1 :}
\FunctionTok{round}\NormalTok{(}
  \FunctionTok{c}\NormalTok{(prev\_reg\_lin, prev\_reg\_morc, prev\_reg\_loc, prev\_ssm), }
  \DecValTok{2}\NormalTok{)}
\end{Highlighting}
\end{Shaded}

\begin{verbatim}
[1] 0.43 0.30 0.29 0.26
\end{verbatim}

Les prévisions entre les différentes méthodes sont proches car sur ce
dernier trimestre le climat des affaires a peu évolué (différence de
0,4).

\section{Comparaison générale}\label{sec-comp-generales}

Dans cette section nous effectuons une comparaison plus détaillée des
différentes méthodes utilisées. Pour cela, nous utilisons 28 modèles de
prévisions des taux de croissance trimestriels de l'industrie
manufacturière et de ses principales sous-branches (voir annexe
\ref{sec-an-graph} pour les graphiques de ces variables). Les modèles
sont estimés entre 1990 et 2019 en utilisant des données issues des
enquêtes de conjoncture de l'Insee et de la Banque de France, ainsi que
l'Indice de Production Industrielle des branches étudiées. Toutes les
séries utilisées sont disponibles sous \faIcon{r-project} dans la base
de donnée \texttt{tvCoef::manufacturing}. Parmi ces 28 modèles, la
procédure de Bai et Perron détecte au moins une rupture sur 14 modèles
et le test de Hansen, utilisé avec un seuil de 5~\%, conclut à la
présence de coefficients mobiles dans 5 modèles
(table~\ref{tbl-nb-models}). Cela permet donc également de comparer les
résultats de la régression locale et de la régression avec coefficients
stochastiques (modélisation espace-état) lorsque les coefficients ne
sont pas considérés comme fixes par les tests étudiés. Dans la suite,
nous considérerons qu'un modèle n'a pas de rupture lorsque le test de
Bai et Perron n'en détecte aucune : dans ce cas, la régression par
morceaux donne le même résultat que la régression linéaire.

\begin{longtable}{l|ccc}

\caption{\label{tbl-nb-models}Nombre de modèles étudiés par branche
d'activité.}

\tabularnewline

\toprule
\multicolumn{1}{l}{} &  & \multicolumn{2}{c}{Avec rupture} \\ 
\cmidrule(lr){3-4}
\multicolumn{1}{l}{} & Total & Bai et Perron & Hansen \\ 
\midrule\addlinespace[2.5pt]
Industrie manufacturière (C) & 5 & 1 & 0 \\ 
Agro–alimentaire (C1) & 5 & 0 & 0 \\ 
Biens d'équipement (C3) & 6 & 4 & 2 \\ 
Matériels de transport (C4) & 6 & 4 & 0 \\ 
Autres industries (C5) & 6 & 5 & 3 \\ 
\midrule 
\midrule 
Total & 28 & 14 & 5 \\ 
\bottomrule

\end{longtable}

{\footnotesize\raggedright

\textbf{Lecture} : dans la branche des biens d'équipement (C3), 6
modèles sont étudiés. Dans 4 de ces modèles la procédure de Bai et
Perron conclut à la présence d'au moins une rupture et le test joint
d'Hansen conclut que les coefficients sont mobiles dans 2 de ces
modèles.\\
\textbf{Sources} : Insee (comptes trimestriels, indices de production
industrielle et enquêtes de conjoncture téléchargés le 15 mars 2024),
Banque de France (enquêtes de conjoncture téléchargées le 15 mars 2024),
calculs de l'auteur.

}

Pour l'estimation des modèles de régression par morceaux, nous limitons
le nombre maximal de ruptures à 2. La dernière rupture est détectée en
2011T4, pour deux modèles de la branche des autres industries (C5). Les
prévisions hors échantillon sont calculées après 2013 afin d'éviter les
fortes erreurs autour des ruptures. Pour la régression locale et la
régression avec coefficients stochastiques (modélisation espace-état),
les modèles sont estimés avec les paramètres par défaut. Leur qualité
prédictive pourrait même être améliorée avec une optimisation du modèle
(par exemple en ne fixant pas les coefficients des indicatrices), au
contraire de la régression linéaire ou par morceaux.

Pour comparer les prévisions, nous utilisons la racine carrée de
l'erreur quadratique moyenne --- \emph{Root-mean-square error} (RMSE).
Afin de comparer les résultats entre les différentes branches, nous
normalisons les RMSE par celles calculées par la régression linéaire,
c'est ce qui est montré dans la table~\ref{tbl-error-table}. Dans
l'ensemble ce sont les régressions avec coefficients stochastiques
(modélisation espace-état) qui donnent les meilleurs résultats. C'est
d'abord le cas pour les modèles où une rupture a été détectée. En effet
dans l'échantillon, les performances sont proches entre les différentes
méthodes (amélioration de la qualité prédictive moyenne d'environ 10 \%
par rapport à la régression linéaire). Et hors échantillon, les
résultats sont également améliorés avec la régression avec coefficients
stochastiques pour la majorité des modèles (en moyenne de 5~\% et au
maximum de 13~\%) mais sont dégradés pour 3 des 14 modèles (d'au plus
7~\%). Alors que pour la régression locale, les résultats sont
identiques ou dégradés dans la majorité des cas (9 modèles sur 14) ; et
pour la régression par morceaux, même si pour 7 séries les résultats
sont améliorés, cette amélioration semble moins forte qu'avec la
régression avec coefficients stochastiques (au plus 9~\%).

C'est ensuite, étonnamment, aussi le cas lorsqu'aucune rupture n'est
détectée, puisque les régression avec coefficients stochastiques
permettent également d'améliorer les résultats : dans l'échantillon les
erreurs de prévision sont réduites d'en moyenne 9~\% et d'au plus 46~\%
(contre 2 \% en moyenne et d'au plus 15~\% pour la régression locale).
En temps réel elles sont améliorées pour huit modèles (d'au plus 15~\%)
et ne sont que légèrement dégradées pour trois autres modèles. Pour la
régression locale, la performance hors échantillon est toujours
dégradée. Une partie de l'instabilité en temps réel provient du fait
qu'aucune optimisation n'est faite dans la spécification des modèles.
Toutefois ces résultats suggèrent que les tests ici présentés pour
tester la constance des coefficients ne sont pas toujours pertinents.
Ainsi, \textcite{abs2006} proposent une procédure de tests fondée sur la
modélisation espace-état (en testant la significativité de la variance
des coefficients estimés).

\begin{longtable}{l|cccccc}

\caption{\label{tbl-error-table}Racine carrée de l'erreur quadratique
moyenne (RMSE) rapportée à celle des modèles régression linéaire.}

\tabularnewline

\toprule
\multicolumn{1}{l}{} &  &  &  & \multicolumn{3}{c}{Séries dont RMSE} \\ 
\cmidrule(lr){5-7}
\multicolumn{1}{l}{} & Moyenne & Min & Max & < 1 & = 1 & > 1 \\ 
\midrule\addlinespace[2.5pt]
\multicolumn{7}{l}{Sans rupture - Dans l'échantillon} \\ 
\midrule\addlinespace[2.5pt]
Rég. par morceaux & $1,00$ & $1,00$ & $1,00$ & $0$ & $14$ & $0$ \\ 
Rég. locale & $0,98$ & $0,85$ & $1,00$ & $5$ & $9$ & $0$ \\ 
Coef. stochastiques (espace-état) & $0,91$ & $0,54$ & $1,00$ & $10$ & $4$ & $0$ \\ 
\midrule\addlinespace[2.5pt]
\multicolumn{7}{l}{Sans rupture - Hors échantillon} \\ 
\midrule\addlinespace[2.5pt]
Rég. par morceaux & $1,00$ & $1,00$ & $1,00$ & $0$ & $14$ & $0$ \\ 
Rég. locale & $1,03$ & $1,00$ & $1,09$ & $0$ & $4$ & $10$ \\ 
Coef. stochastiques (espace-état) & $0,98$ & $0,85$ & $1,02$ & $8$ & $3$ & $3$ \\ 
\midrule\addlinespace[2.5pt]
\multicolumn{7}{l}{Avec rupture - Dans l'échantillon} \\ 
\midrule\addlinespace[2.5pt]
Rég. par morceaux & $0,89$ & $0,70$ & $0,98$ & $14$ & $0$ & $0$ \\ 
Rég. locale & $0,90$ & $0,74$ & $0,99$ & $14$ & $0$ & $0$ \\ 
Coef. stochastiques (espace-état) & $0,88$ & $0,68$ & $0,98$ & $14$ & $0$ & $0$ \\ 
\midrule\addlinespace[2.5pt]
\multicolumn{7}{l}{Avec rupture - Hors échantillon} \\ 
\midrule\addlinespace[2.5pt]
Rég. par morceaux & $1,02$ & $0,91$ & $1,11$ & $7$ & $0$ & $7$ \\ 
Rég. locale & $1,04$ & $0,96$ & $1,19$ & $5$ & $1$ & $8$ \\ 
Coef. stochastiques (espace-état) & $0,95$ & $0,87$ & $1,07$ & $10$ & $1$ & $3$ \\ 
\bottomrule

\end{longtable}

{\footnotesize\raggedright

\textbf{Note} : les modèles sans rupture sont ceux où aucune rupture
n'est détectée par la procédure de Bai et Perron, la régression par
morceaux coïncide alors avec la régression linéaire.\\
\textbf{Lecture} : pour les prévisions hors échantillon la régression
avec coefficients stochastiques (modélisation espace-état) permet, par
rapport à la régression linéaire, de réduire la RMSE d'en moyenne de
5~\% pour les modèles avec rupture et de 2~\% pour les modèles sans
rupture. Huit modèles sans rupture sont améliorés avec la régression
avec coefficients stochastiques (avec une réduction maximale de la RMSE
de 15~\%) et les erreurs de prévision sont augmentées pour trois modèles
(avec une hausse maximale de 2~\%).\\
\textbf{Sources} : Insee (comptes trimestriels, indices de production
industrielle et enquêtes de conjoncture téléchargés le 15 mars 2024),
Banque de France (enquêtes de conjoncture téléchargées le 15 mars 2024),
calculs de l'auteur.

}

\section{Conclusion}\label{conclusion}

En conclusion, cette étude montre comment, à partir d'un modèle de
régression linéaire, l'hypothèse de constance des coefficients peut être
testée et comment relâcher cette hypothèse en implémentant des modèles
de régression par morceaux, de régression mobile et de régression avec
coefficients stochastiques (modélisation espace-état). Cette
implémentation est facilitée grâce au package \texttt{tvCoef}
(\url{https://github.com/InseeFrLab/tvCoef}) qui accompagne cette étude
et tous les codes associés sont disponibles sous
\url{https://github.com/InseeFrLab/DT-tvcoef}.

Lorsque les tests classiques indiquent une non-constance des
coefficients, ces trois méthodes permettent de réduire les erreurs de
prévision dans l'échantillon (lorsque les coefficients sont estimés sur
l'ensemble des données). Toutefois, ces trois méthodes reposent sur des
hypothèses qui peuvent conduire à de fortes instabilités, notamment si
elles sont utilisées naïvement lors des exercices de prévision en temps
réel (hors échantillon).\\
La régression par morceaux suppose la connaissance de dates de ruptures
: même si des procédures existent pour leur détection automatique
\autocite{bai2003computation}, les instabilités autour de celles-ci font
qu'il est préférable de s'appuyer sur un raisonnement économique. En
effet, si rupture brutale il y a, elle doit pouvoir s'expliquer et le
statisticien devrait pouvoir l'expliquer.\\
Le paramètre principal de la régression locale est la fenêtre, qui
permet de jouer sur la sensibilité des estimations aux observations
lointaines. Même s'il existe également des procédures de sélection
automatique, leurs instabilités conduisent à des erreurs de prévision
plus élevées que la régression linéaire lors des exercices de prévisions
en temps réel.\\
Enfin, dans les régressions avec coefficients stochastiques
(modélisation espace-état), des instabilités numériques d'optimisation
peuvent conduire à une volatilité dans l'estimation des variances des
coefficients (qui déterminent si le coefficient varie ou non dans le
temps et à quelle vitesse). C'est toutefois la méthode qui donne les
meilleurs résultats et qui permet dans la majorité des cas de réduire
les erreurs de prévision par rapport à la régression linéaire, même
lorsque les tests classiques indiquent une constance des coefficients !

Même si ces méthodes permettent, par rapport à la régression linéaire,
d'améliorer la qualité des prévisions, elles n'ont pas vocation à
remplacer les modèles existants mais plutôt à les compléter. En effet,
même si dans la majorité des cas les méthodes étudiées permettent de
réduire les erreurs de prévision, cela peut ne pas être le cas sur tous
les trimestres. D'une part la combinaison de prévisions issues de
différents modèles permet généralement d'obtenir une prévision finale
plus précise \autocite[voir par exemple][ pour une revue de
littérature]{WANG20231518} ; d'autre part, l'interprétation économique
et les hypothèses sous-jacentes sont différentes entre chaque modèle :
l'analyse faite de la prévision dépend donc également de la conjoncture
récente.

Cette étude pourrait être étendue de plusieurs manières. Tout d'abord,
les méthodes ici présentées pourraient être améliorées. Par exemple,
pour la régression locale et la régression avec coefficients
stochastiques, nous supposons que les paramètres évoluent à la même
vitesse au cours de toute la période d'estimation (fenêtre fixe et
variance fixée). Toutefois, autour des périodes de crises (comme le
COVID-19), il pourrait être pertinent d'ajouter plus de flexibilité à
l'évolution des coefficients (en réduisant la fenêtre ou en effectuant
un choc sur la variance) : cela ajouterait plus de variabilité dans les
estimations mais pourrait permettre de mieux prendre en compte les
changements structurels.\\
Ensuite, d'autres méthodes d'estimations pourraient être utilisées pour
modéliser l'évolution dans le temps des coefficients. Par exemple
\textcite{melard} modélise des variations déterministes des coefficients
(plutôt que stochastiques comme pour les modèles espace-état).\\
Enfin, les modèles auraient également pu être comparés aux modèles à
seuil et modèles à changement de régime markoviens \autocite[pour une
revue bibliographique de ces modèles, voir par
exemple][]{PETROPOULOS2022705} qui peuvent se voir comme des cas
particuliers des méthodes étudiées. En effet, dans les modèles à seuil
la rupture est brutale et dépend du niveau d'une variable exogène et
dans les modèles à changement de régime markoviens, les coefficients
dépendent d'une variable inobservée modélisant la position de l'économie
dans le cycle : la rupture est donc brutale et dépend d'une variable
externe (comme dans la régression locale). \emph{In fine}, le choix
entre toutes ces méthodes se fait surtout sur les hypothèses économiques
que l'on souhaite modéliser.

\newpage
\appendix

\section{\texorpdfstring{Installation de
\texttt{tvCoef}}{Installation de tvCoef}}\label{installation-de-tvcoef}

Pour utiliser \texttt{tvCoef}, il faut il faut avoir la version 17 de
Java SE (ou une version supérieure).

Pour savoir quelle version de Java est utilisée par R, utiliser le code
suivant :

\begin{Shaded}
\begin{Highlighting}[]
\FunctionTok{library}\NormalTok{(rJava)}
\FunctionTok{.jinit}\NormalTok{()}
\FunctionTok{.jcall}\NormalTok{(}\StringTok{"java/lang/System"}\NormalTok{, }\StringTok{"S"}\NormalTok{, }\StringTok{"getProperty"}\NormalTok{, }\StringTok{"java.runtime.version"}\NormalTok{)}
\end{Highlighting}
\end{Shaded}

Si le résultat n'est pas sous la forme \texttt{"17xxxx"} c'est que vous
n'avez pas Java 17 !

Si l'on a pas cette version d'installée et que l'on n'a pas les droits
d'administrateur pour installer Java, une solution est d'installer une
version portable de Java, par exemple installer une version portable à
partir des liens suivants :

\begin{itemize}
\item
  \href{https://www.azul.com/downloads/\#zulu}{Zulu JDK}
\item
  \href{https://adoptopenjdk.net/}{AdoptOpenJDK}
\item
  \href{https://aws.amazon.com/corretto/}{Amazon Corretto}
\end{itemize}

Pour installer une version portable de java, télécharger par exemple le
fichier \texttt{Windows\ 10\ x64\ Java\ Development\ Kit} disponible sur
\url{https://jdk.java.net/java-se-ri/17}, le dézipper et le mettre par
exemple sous \texttt{"D:/Programmes/jdk-17"}.

Pour configurer R avec une version portable de Java, trois solutions~:

\begin{enumerate}
\def\labelenumi{\arabic{enumi}.}
\tightlist
\item
  Avant \textbf{tout chargement de package nécessitant Java
  (\texttt{rJava}\ldots)} (si vous avez lancé le code précédent,
  relancez donc R) :
\end{enumerate}

\begin{Shaded}
\begin{Highlighting}[]
\CommentTok{\# Si la version portable est installée sous D:/Programmes/jdk{-}17}
\FunctionTok{Sys.setenv}\NormalTok{(}\AttributeTok{JAVA\_HOME=}\StringTok{\textquotesingle{}D:/Programmes/jdk{-}17\textquotesingle{}}\NormalTok{)}
\end{Highlighting}
\end{Shaded}

\begin{enumerate}
\def\labelenumi{\arabic{enumi}.}
\setcounter{enumi}{1}
\tightlist
\item
  Pour éviter de faire cette manipulation à chaque fois que l'on relance
  R, deux solutions~:
\end{enumerate}

\begin{enumerate}
\def\labelenumi{\alph{enumi}.}
\item
  Modifier le \texttt{JAVA\_HOME} dans les variables d'environnement de
  Windows (voir
  \url{https://confluence.atlassian.com/doc/setting-the-java_home-variable-in-windows-8895.html}).
\item
  Modifier le \texttt{.Renviron} : depuis R lancer le code
  \VERB|\FunctionTok{file.edit}\NormalTok{(}\StringTok{"\textasciitilde{}/.Renviron"}\NormalTok{)}|,
  ajouter dans le fichier le chemin vers la version portable de Java
  comme précédemment
  (\texttt{JAVA\_HOME=\textquotesingle{}D:/Programmes/jdk-17\textquotesingle{}}),
  sauvegarder et relancer R.
\end{enumerate}

Il reste maintenant à installer les packages :

\begin{Shaded}
\begin{Highlighting}[]
\CommentTok{\# Nécessaire pour installer rjd3sts}
\NormalTok{remotes}\SpecialCharTok{::}\FunctionTok{install\_github}\NormalTok{(}\StringTok{"rjdverse/rjd3toolkit"}\NormalTok{)}
\CommentTok{\# Pour installer rjd3sts (modèles espace{-}état)}
\NormalTok{remotes}\SpecialCharTok{::}\FunctionTok{install\_github}\NormalTok{(}\StringTok{"rjdverse/rjd3sts"}\NormalTok{)}
\NormalTok{remotes}\SpecialCharTok{::}\FunctionTok{install\_github}\NormalTok{(}\StringTok{"InseeFrLab/tvCoef"}\NormalTok{)}
\end{Highlighting}
\end{Shaded}

Si vous utilisez un ordinateur professionnel, si c'est nécessaire,
pensez à configurer le proxy pour que ces commandes puissent fonctionner
(voir
\url{https://www.book.utilitr.org/01_r_insee/fiche-personnaliser-r\#le-fichier-.renviron}).
Pour cela vous pouvez utiliser \texttt{curl::ie\_get\_proxy\_for\_url()}
pour récupérer l'adresse du proxy et ajouter deux variable
\texttt{http\_proxy} et \texttt{https\_proxy} dans les variables
d'environnement (comme précédemment).

\newpage

\section{Annexe graphiques}\label{sec-an-graph}

\begin{figure}

\caption{\label{fig-graph-pib}Taux de croissance trimestriel du PIB
français (variable \texttt{tvCoef::gdp{[},\ "growth\_gdp"{]}}).}

\centering{

\includegraphics{DT-tvcoef_files/figure-pdf/graph-pib-1.pdf}

\footnotesize\raggedright

\textbf{Lecture} : le premier graphique représente la série observée
jusqu'au 2023T4 et le second graphique correspond à un zoom sur la
période pre-covid (avant 2020).\\
\textbf{Source} : Insee (données téléchargées le 15 mars 2024).

}

\end{figure}%

\begin{figure}

\caption{\label{fig-graph-climatfrm1}Climat des affaires France en
niveau au premier mois de chaque trimestre (variable
\texttt{tvCoef::gdp{[},\ "bc\_fr\_m1"{]}}).}

\centering{

\includegraphics{DT-tvcoef_files/figure-pdf/graph-climatfr-m1-1.pdf}

\footnotesize\raggedright

\textbf{Lecture} : le premier graphique représente la série observée
jusqu'au 2023T4 et le second graphique correspond à un zoom sur la
période pre-covid (avant 2020).\\
\textbf{Source} : Insee (données téléchargées le 15 mars 2024).

}

\end{figure}%

\begin{figure}

\caption{\label{fig-graph-diff-climatfrm1}Différenciation trimestrielle
du climat des affaires France au premier mois de chaque trimestre
(variable \texttt{tvCoef::gdp{[},\ "diff\_bc\_fr\_m1"{]}}).}

\centering{

\includegraphics{DT-tvcoef_files/figure-pdf/graph-diff-climatfr-m1-1.pdf}

\footnotesize\raggedright

\textbf{Lecture} : le premier graphique représente la série observée
jusqu'au 2023T4 et le second graphique correspond à un zoom sur la
période pre-covid (avant 2020).\\
\textbf{Source} : Insee (données téléchargées le 15 mars 2024).

}

\end{figure}%

\begin{figure}

\caption{\label{fig-graph-manuf}Taux de croissance trimestriel de la
production manufacturière (variable
\texttt{tvCoef::manufacturing{[},\ "manuf\_prod"{]}}).}

\centering{

\includegraphics{DT-tvcoef_files/figure-pdf/graph-manuf-1.pdf}

\footnotesize\raggedright

\textbf{Lecture} : le premier graphique représente la série observée
jusqu'au 2023T4 et le second graphique correspond à un zoom sur la
période pre-covid (avant 2020).\\
\textbf{Source} : Insee (données téléchargées le 15 mars 2024).

}

\end{figure}%

\begin{figure}

\caption{\label{fig-graph-prodc1}Taux de croissance trimestriel de la
production dans la branche agro--alimentaire (C1) (variable
\texttt{tvCoef::manufacturing{[},\ "prod\_c1"{]}}).}

\centering{

\includegraphics{DT-tvcoef_files/figure-pdf/graph-prod-c1-1.pdf}

\footnotesize\raggedright

\textbf{Lecture} : le premier graphique représente la série observée
jusqu'au 2023T4 et le second graphique correspond à un zoom sur la
période pre-covid (avant 2020).\\
\textbf{Source} : Insee (données téléchargées le 15 mars 2024).

}

\end{figure}%

\begin{figure}

\caption{\label{fig-graph-prodc3}Taux de croissance trimestriel de la
production dans la branche biens d'équipement (C3) (variable
\texttt{tvCoef::manufacturing{[},\ "prod\_c3"{]}}).}

\centering{

\includegraphics{DT-tvcoef_files/figure-pdf/graph-prod-c3-1.pdf}

\footnotesize\raggedright

\textbf{Lecture} : le premier graphique représente la série observée
jusqu'au 2023T4 et le second graphique correspond à un zoom sur la
période pre-covid (avant 2020).\\
\textbf{Source} : Insee (données téléchargées le 15 mars 2024).

}

\end{figure}%

\begin{figure}

\caption{\label{fig-graph-prodc4}Taux de croissance trimestriel de la
production dans la branche matériels de transport (C4) (variable
\texttt{tvCoef::manufacturing{[},\ "prod\_c4"{]}}).}

\centering{

\includegraphics{DT-tvcoef_files/figure-pdf/graph-prod-c4-1.pdf}

\footnotesize\raggedright

\textbf{Lecture} : le premier graphique représente la série observée
jusqu'au 2023T4 et le second graphique correspond à un zoom sur la
période pre-covid (avant 2020).\\
\textbf{Source} : Insee (données téléchargées le 15 mars 2024).

}

\end{figure}%

\begin{figure}

\caption{\label{fig-graph-prodc5}Taux de croissance trimestriel de la
production dans la branche autres industries (C5) (variable
\texttt{tvCoef::manufacturing{[},\ "prod\_c5"{]}}).}

\centering{

\includegraphics{DT-tvcoef_files/figure-pdf/graph-prod-c5-1.pdf}

\footnotesize\raggedright

\textbf{Lecture} : le premier graphique représente la série observée
jusqu'au 2023T4 et le second graphique correspond à un zoom sur la
période pre-covid (avant 2020).\\
\textbf{Source} : Insee (données téléchargées le 15 mars 2024).

}

\end{figure}%

\clearpage

\section*{Bibliographie}\label{bibliographie}
\addcontentsline{toc}{section}{Bibliographie}

\printbibliography[heading=none]




\end{document}
